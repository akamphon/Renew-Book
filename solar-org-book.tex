% Created 2018-07-16 Mon 23:51
% Intended LaTeX compiler: xelatex
\documentclass[11pt]{article}
\usepackage{graphicx}
\usepackage{grffile}
\usepackage{longtable}
\usepackage{wrapfig}
\usepackage{rotating}
\usepackage[normalem]{ulem}
\usepackage{amsmath}
\usepackage{textcomp}
\usepackage{amssymb}
\usepackage{capt-of}
\usepackage{hyperref}
\usepackage{fontspec}
\usepackage{polyglossia}
\setdefaultlanguage{thai}
\usepackage{xltxtra}
\usepackage[svgnames]{xcolor}
\XeTeXlinebreaklocale "th_TH"
\XeTeXlinebreakskip = 0pt plus 1pt
\newfontfamily{\thaifont}[Scale=1.23]{TH Sarabun New}
\usepackage[parfill]{parskip}
\usepackage{tikz}
\usetikzlibrary{arrows,calc,decorations,shapes,shapes.misc,positioning,decorations.pathmorphing,patterns}
\usepackage{circuitikz}
\usepackage{pgfplots}
\pgfplotsset{compat=1.16}
\usepackage{newtxmath}
\author{สัปปินันทน์ เอกอำพน}
\date{2018-07-16}
\title{พลังงานทดแทนในประเทศไทย}
\hypersetup{
 pdfauthor={สัปปินันทน์ เอกอำพน},
 pdftitle={พลังงานทดแทนในประเทศไทย},
 pdfkeywords={},
 pdfsubject={},
 pdfcreator={Emacs 27.0.50 (Org mode 9.1.13)}, 
 pdflang={English}}
\begin{document}

\maketitle
\tableofcontents

\maketitle

\section{คำนำ}
\label{sec:orgab46bd3}

ตำราเล่มนี้ถูกเขียนขึ้นเพื่อใช้ประกอบการเรียนการสอนเกี่ยวกับการใช้พลังงานแสงอาทิตย์สำหรับนักศึกษาปี ๓ - ๔ และสำหรับบุคคลทั่วไปที่มีความสนใจทางด้านดังกล่าว โดยที่แม้เนื้อหาบางส่วนจะมีคณิตศาสตร์ชั้นสูงเพื่อช่วยในการแสดงความสัมพันธ์ระหว่างตัวแปร แต่ความตั้งใจหลักของผู้เขียนต้องการจะให้ผู้ที่มีความสนใจและมีพื้นฐานคณิตศาสตร์ระดับมัธยมปลายควรจะสามารถอ่านแล้วเข้าใจได้ ทั้งนี้เนื่องจากผู้เขียนเล็งเห็นความสำคัญของการสร้างความเข้าใจพื้นฐานเรื่องของพลังงานแสงอาทิตย์ รวมถึงเทคโนโลยีต่างๆที่จะนำไประยุกต์ใช้เพื่อกักเก็บ แปลง หรือนำพลังงานนี้ไปใช้ เพื่อให้ผู้อ่านจะได้มีความเข้าใจที่ถูกต้อง มีพื้นฐานความรู้ที่เหมาะสมในการทำงานในเทคโนโลยีพลังงานสะอาดในอนาคต หรือแม้แต่สามารถทำความเข้าใจและคำนึงถึงความเหมาะสมของนโยบายหรือโครงการที่เกี่ยวกับพลังงานแสงอาทิตย์ได้โดยไม่เชื่อเพียงคำโฆษณาหรืออวดอ้างที่อาจจะเกินความเป็นจริงในหลายครั้ง

ผู้เขียนหวังว่าข้อมูลที่ได้รับการรวบรวมไว้ในตำราเล่มนี้จะเป็นประโยชน์ต่อผู้อ่านในวงกว้าง มิใช่เฉพาะระดับนักศึกษาหรือนักวิชาการเท่านั้น อย่างไรก็ดี ถ้าหากผู้อ่านมีความรู้พื้นฐานทางด้านฟิสิกส์พื้นฐาน จะทำให้สามารถเข้าใจเนื้อหาและบทวิเคราะห์ได้ดียิ่งขึ้น รวมถึงสามารถนำความรู้ที่ได้รับนำไปวิเคราะห์ข้อมูลอื่นๆได้ด้วยตนเอง

\section{พลังงานแสงอาทิตย์}
\label{sec:org89dd86a}

เวลาพูดถึงพลังงานแสงอาทิตย์นั้น หลายๆคนอาจจะนึกถึงแดดร้อนๆในช่วงเดือนมีนาคมหรือเมษายน แต่จริงๆแล้วจะรู้ไหมว่าพลังงานที่มีอยู่ในแสงอาทิตย์นั้นประกอบด้วยหลายส่วน การจะตักตวงพลังงานแสงอาทิตย์มาใช้ให้ได้เต็มที่นั้น จำเป็นที่เราจะต้องมีความเข้าใจถึงส่วนประกอบเหล่านี้

เนื่องจากพลังงานแสงอาทิตย์นั้นเป็นพลังงานที่เกิดขึ้นมาจากการแผ่รังสีของดวงอาทิตย์ออกมาในรูปของคลื่นแม่เหล็กไฟฟ้าในช่วงคลื่นต่างๆ ดังนั้นเราควรจะเริ่มทำความเข้าใจกับการแผ่รังสีของวัตถุดำก่อน

\subsection{การแผ่รังสีของวัตถุดำ (Blackbody Radiation)}
\label{sec:org9574d41}

การแผ่รังสีของวัตถุดำเกิดจากการแผ่รังสีคลื่นแม่เหล็กไฟฟ้าจากความร้อนของวัตถุซึ่งอยู่ในสภาวะสมดุลทางอุณหพลศาสตร์กับสิ่งแวดล้อม ซึ่งช่วงความถี่และความเข้มข้นของคลื่นต่างๆนั้นขึ้นอยู่กับอุณหภูมิของวัตถุดังกล่าว อย่างไรก็ดี ในความเป็นจริงแล้วไม่มีวัตถุได้ที่มีการแผ่รังสีเหมือนวัตถุดำแท้จริง โดนเฉพาะอย่างยิ่งดาวฤกษ์อย่างพระอาทิตย์นั้นก็ไม่ได้อยู่ในสภาวะสมดุลกับสิ่งแวดล้อม แต่ความเข้าใจเรื่องของการแผ่รังสีนี้ก็สามารถนำมาใช้ทำความเข้าใจส่วนประกอบของแสงอาทิตย์ได้

ยกตัวอย่างเช่น ในวัตถุที่มีอุณหภูมิต่ำนั้น ในห้องมืดจะมองเห็นเป็นสีดำเนื่องจากช่วงคลื่นที่แผ่ออกมาเป็นช่วงอินฟราเรดซึ่งมองด้วยตาเปล่าไม่เห็น เมื่ออุณหภูมิสูงขึ้นถึงราว 500\(^{\circ}\) C การแผ่รังสีเริ่มเข้าอยู่ในช่วงความถี่ที่ตามองเห็น (visible spectrum) และจะเริ่มมีสีแดง เมื่ออุณหภูมิสูงมากจะออกเป็นสีฟ้าขาว เมื่อวัตถุมีการแผ่รังสีเป็นสีขาว แสดงว่ามีการแผ่รังสีบางส่วนออกมาเป็นรังสีอัลตราไวโอเลต

ดวงอาทิตย์ซึ่งมีอุณหภูมิที่ผิวประมาณ 5800 K นั้น มีการแผ่รังสีออกมามากที่สุดในช่วงคลื่นแสงและอินฟราเรด และมีจำนวนอีกเล็กน้อยในช่วงอัลตราไวโอเลต

\subsection{ทิศทางของแสงอาทิตย์}
\label{sec:orgcf41d51}

เนื่องจากดวงอาทิตย์เคลื่อนที่อยู่ตลอดเวลา และพลังงานของแสงอาทิตย์ที่ตกกระทบลงบนพื้นที่หนึ่งๆขึ้นอยู่กับความเข้มข้นของแสงและมุมตกกระทบ เพื่อจะเพิ่มพลังงานแสงอาทิตย์ที่ได้รับ เราสามารถออกแบบอุปกรณ์ให้มีความสามารถในการติดตามดวงอาทิตย์ (solar tracking) ซึ่งในปัจจุบันมีเทคโนโลยีหลายวิธีที่ใช้ในการติดตาม ซึ่งแบ่งได้เป็น 2 ประเภทใหญ่

\subsubsection{การติดตามแบบใช้พลังงาน}
\label{sec:org79d6725}

การติดตามดวงอาทิตย์แบบใช้พลังงานหรือที่เรียกว่า Active Tracking นั้นเป็นการใช้ระบบ Feedback Loop โดยใช้ตัวรับแสงเพื่อช่วยในการบอกตำแหน่งของดวงอาทิตย์ที่ประเมินผล แล้วส่งสัญญาณให้กับระบบควบคุมให้เคลื่อนที่ไปยังตำแหน่งที่ต้องการ

\subsubsection{การติดตามแบบไม่ใช้พลังงาน}
\label{sec:org7c249ef}

\section{เซลล์แสงอาทิตย์}
\label{sec:org53927b1}

เซลล์แสงอาทิตย์ (Solar cell หรือ Photovoltaic cell) เป็นอุปกรณ์ที่สามารถแปลงพลังงานจากคลื่นแม่เหล็กไฟฟ้าในแสงอาทิตย์ให้เป็นพลังงานไฟฟ้าได้โดยตรงโดยใช้ปรากฏการณ์โฟโตโวลตาอิก (Photovoltaic effect) ปรากฏการณ์นี้เกิดขึ้นจากการเคลื่อนไหวของอิเลกตรอนในเซลล์แสงอาทิตย์เมื่อได้ดูดซับพลังงานแสงอาทิตย์ ซึ่งทำให้เกิดกระแสไฟฟ้าซึ่งสามารถนำไปใช้ให้เกิดประโยชน์ได้

จริงๆแล้วปรากฏการณ์โฟโตโวลตาอิกนั้นสามารถเกิดขึ้นได้ในวัสดุอื่นๆนอกจากเซลล์สุริยะด้วย แต่เนื่องจากการเคลื่อนที่ของอิเลกตรอนจากปรากฏการณ์ดังกล่าวนั้นไม่มีทิศทางหรือแนวโน้มใดๆ จึงทำให้ไม่มีกระแสลัพธ์เกิดขึ้น จำเป็นจะต้องมีวิธีบังคับการไหลของอิเลกตรอนเพื่อให้เกิดกระแสได้ นั่นเป็นสาเหตุที่เซลล์สุริยะจำเป็นจะต้องมีการออกแบบวงจรพิเศษ

\subsection{หลักการทำงานของเซลล์แสงอาทิตย์}
\label{sec:org67c9a44}

ในเซลล์สุริยะนั้น ระบบวงจรที่จะบังคับทิศทางการไหลของอิเลกตรอนที่เกิดจากปรากฏการณ์โฟโตโวทาอิกคือ P-N junction ซึ่งเป็นการเชื่อมต่อระหว่างสารกึ่งตัวนำประเภทบวก (P-type) กับประเภทลบ (N-type) โดยที่สาร P-type นั้นมีหลุมอิเลกตรอนเนื่องมาจากการ dope สารที่ขาดอิเลกตรอนลงไปในซิลิกอน ส่วนสาร N-type นั้นมีอิเลกตรอนอิสระเนื่องจากการ dope สารที่มีอิเลกตรอนอิสระลงไป เมื่อนำสารทั้งสองแบบมาเชื่อมต่อกัน หลุมอิเลกตรอนและอิเลกตรอนอิสระเคลื่อนที่เข้าหากันทำให้เกิด \textbf{Depletion Zone} ซึ่งป้องกันการไหลของอิเลกตรอนอีก เมื่อแสงอาทิตย์ตกกระทบ อิเลกตรอนอิสระและหลุมอิเลกตรอนที่เกิดขึ้นจึงถูกบังคับให้ไหลผ่านความต้านทานภายนอกซึ่งทำให้เกิดกระแสไฟฟ้าขึ้น

ปริมาณกระแสที่เซลล์แสงอาทิตย์สร้างขึ้นได้นั้นขึ้นอยู่กับปัจจัยหลายประการ เช่น ประสิทธิภาพของ P-N junction ในการป้องกันกระแสย้อนกลับ และประสิทธิภาพของวัสดุเซลล์ในการสร้างอิเลกตรอนเมื่อมีแสงอาทิตย์ตกกระทบ ซึ่งระบบเซลล์แสงอาทิตย์สามารถเขียนแทนได้ด้วยวงจรเทียบเท่าได้โดยไดโอดและความต้านทานภายในดังรูป \ref{fig: equiv circuit solar cell}

\begin{figure}[h]
  \centering
  \ctikzset{bipoles/length=1cm}
  \begin{tikzpicture}
    \draw[color=Black] (0,0) to [I,l^=$i_{PV}$] ++(90:3) to [short] ++(0:1) to [Do, i=$i_D$] ++(-90:3) to [short] ++(180:1);
    % \draw[color=Black] (1,3) to [short] ++(0:1) to [R,l^=$R_{SH}$,i=$I_{SH}$] ++(-90:3) to [short] ++(180:1);
    \draw[color=Black] (1,3) to [R,l^=$R_s$, i=$i$, -*] ++(0:3);
    \draw[color=Black] (1,0) to [short,-*] ++(0:3);
    \node at (4,1.5) {$V_L$};
  \end{tikzpicture}
  \caption{วงจรเทียบเท่าของเซลล์แสงอาทิตย์}
  \label{fig: equiv circuit solar cell}
\end{figure}

จากวงจรเทียบเท่าดังกล่าว สามารถเขียนสมการแสดงปริมาณกระแสที่เซลล์สุริยะได้ว่า กระแสที่ไหลผ่านไปที่โหลดภายนอกเท่ากับกระแสที่เซลล์สุริยะสร้างได้ลบด้วยกระแสที่ไหลย้อนผ่าน P-N junction

$$ i = i_{PV} - i_D $$

ปริมาณกระแสที่ไหลผ่าน P-N junction ขึ้นอยู่กับอุณหภูมิ (\(T\)) และความต่างศักย์ของโหลดภายนอก (\(V\)) โดยสามารถเขียนเป็นสมการได้ดังนี้

\begin{equation}
  \label{eq: pn current}
  i_D = i_0 \left[ exp \left( \frac{eV}{kT} \right) - 1 \right]
\end{equation}

เมื่อแทนสมการ \ref{eq: pn current} ลงในสมการ \ref{eq: solar cell current density} จะได้สมการ

\begin{equation}
  \label{eq: solar cell current}
    i = i_{PV} - i_0\left[exp \left( \frac{eV}{kT} \right) - 1 \right]
\end{equation}

โดยที่ \(i_0\) คือกระแสย้อนกลับอิ่มตัวของ P-N junction, \(i_{PV}\) คือกระแสจากปรากฏการณ์โฟโตโวลทาอิก และ \(i\) คือกระแสที่ผ่านตัวต้านทานภายนอก 

เซลล์สุริยะสามารถผลิตกำลังได้สูงสุดเมื่อ

\begin{gather}
  \label{eq: max solar cell power}
   	P_{out} = i V \nonumber \\
    \frac{dP_{out}}{dV} = 0 \nonumber \\
   	exp \left(\frac{e V_{\max P}}{kT} \right) = \dfrac{1+\dfrac{i_{PV}}{i_0}}{1+ \dfrac{e V_{\max P}}{kT}}
\end{gather}

สังเกตว่าสมการนี้มีค่า \(V_{\max P}\) อยู่ทั้งสองด้าน ไม่สามารถแก้สมการเชิงวิเคราะห์ได้ จำเป็นต้องแก้สมการเชิงตัวเลข

ประสิทธิภาพสูงสุดของแผงเซลล์สุริยะเกิดในตอนที่แผงผลิตกำลังไฟฟ้าสูงสุด ซึ่งเขียนเป็นสมการได้ว่า

\begin{gather}
  \label{eq: solar cell max eff}
  P_{\max} =  \dfrac{V_{\max P} ( i_0 + i_{PV} )}{1 + \dfrac{kT}{e V_{\max P}}} \\
  \eta_{\max} = \eta_{\max P} =  \dfrac{P_{\max}}{I_{in}} = \dfrac{V_{\max P} ( i_0 + i_{PV} )}{I_{in} \left(1 + \dfrac{kT}{e V_{\max P}} \right)}
\end{gather}

\begin{figure}[h]
  \centering
  \begin{tikzpicture}
    \begin{axis} [
      scale only axis,
      % xtick=data,
      xmin=0,xmax=0.5,
      ymin=0,ymax=4,
      xtick distance=0.1,
      xlabel={ความต่างศักย์ $V$ [V]},
      ylabel={กระแส $I$ [A]},
      axis y line*=left,
      ]
      \addlegendimage{empty legend}
      \label{plot_0}
      \addplot [blue, domain=0:0.45, samples=50] {3.5 - 10^(-7)*(exp(\x/0.0259)-1)};
      \label{plot_1}
      \addplot [red, domain=0:0.4, samples=50] {3.5 - 10^(-6)*(exp(\x/0.0259)-1)};
      \label{plot_2}
      \addplot [green, domain=0:0.35, samples=50] {3.5 - 10^(-5)*(exp(\x/0.0259)-1)};
      \label{plot_3}
    \end{axis}
    \begin{axis} [
      scale only axis,
      axis y line*=right,
      axis x line=none,
      xmin=0,xmax=0.5,
      ymin=0,ymax=1.5,
      ylabel={กำลังไฟฟ้า [W]},
      compat=1.3,
      ytick distance=0.3,
      legend style={at={(0.1,0.6)}, anchor=west},
      ]
      \addlegendimage{/pgfplots/refstyle=plot_0}\addlegendentry{\hspace{-6mm}\textbf{กระแส}};
      \addlegendimage{/pgfplots/refstyle=plot_1}\addlegendentry{$10^{-7}$}
      \addlegendimage{/pgfplots/refstyle=plot_2}\addlegendentry{$10^{-6}$}
      \addlegendimage{/pgfplots/refstyle=plot_3}\addlegendentry{$10^{-5}$}
      \addlegendimage{empty legend}
      \addplot [blue, dashed, domain=0:0.45] {\x * (3.5 - 10^(-7)*(exp(\x/0.0259)-1))};
      \addplot [red, dashed, domain=0:0.4] {\x * (3.5 - 10^(-6)*(exp(\x/0.0259)-1))};
      \addplot [green, dashed, domain=0:0.35] {\x * (3.5 - 10^(-5)*(exp(\x/0.0259)-1))};
      \addlegendentry{\hspace{-6mm}\textbf{กำลัง}};
      \addlegendentry{$10^{-7}$};
      \addlegendentry{$10^{-6}$};
      \addlegendentry{$10^{-5}$};
    \end{axis}
  \end{tikzpicture}
  \caption{กราฟแสดงความสัมพันธ์ระหว่างกระแส แรงดันไฟฟ้า และกำลังไฟฟ้าที่ผลิตได้จากในเซลล์แสงอาทิตย์ที่อุณหภูมิ 25$^{\circ}$C}
\end{figure}

\subsection{พลังงานความร้อนแสงอาทิตย์\}}
\label{sec:org0b0e23d}

\subsubsection{การรับความร้อนโดยตรง\}}
\label{sec:org1580a46}


\subsubsection{การรับความร้อนแบบรวมแสง\}}
\label{sec:org2e64dc3}

\subsection{คุณสมบัติของตัวรับแสง\}}
\label{sec:org2c79b56}

\subsection{เทคโนโลยีนำความร้อนไปใช้ต่อ\}}
\label{sec:orgad0d10a}

\section{เทอร์โมอิเล็กทริก (Thermoelectricity)}
\label{sec:org8718e16}

เทอร์โมอิเล็กทริกซิตี้เป็นการแปลงพลังงานโดยตรงจากความร้อนไปเป็นพลังงานไฟฟ้า ซึ่งสารที่สามารถแปลงพลังงานด้วยวิธีนี้ได้เรียกว่าวัสดุเทอร์โมอิเล็กทริก (Thermoelectric materials) ซึ่งเทคโนโลยีนี้มีความน่าสนใจเนื่องจากในปัจจุบันในโลกของเรายังมีแหล่งพลังงานความร้อนราคาถูกอยู่มาก ไม่ว่าจะเป็นแหล่งพลังงานพลังงานแสงอาทิตย์ หรือพลังงานความร้อนเหลือใช้ (Waste heat) จากกระบวนการทางอุตสาหกรรมต่างๆ โดยในการแปลงพลังงานที่เกิดขึ้นนั้นเกิดขึ้นจากปรากฏการณ์เทอร์โมอิเล็กทริก (thermoelectric effect) ซึ่งสามารถแบ่งย่อยออกเป็นปรากฏการณ์ซึ่งเกิดขึ้นพร้อมกัน 3 อย่างดังต่อไปนี้

\subsection{ปรากฏการณ์ซีเบ็ก (Seebeck Effect)}
\label{sec:org5b25314}

เทอร์โมอิเล็กทริกซิตี้เป็นปรากฏการณ์การเกิดศักย์ไฟฟ้าขึ้นบนตัวนำหรือสารกึ่งตัวนำที่มีอุณหภูมิเปลี่ยนไป โดยมีหลักการมาจากการแพร่ (diffusion) ของพาหะของประจุ (charge carrier) ในสารเมื่อได้รับความร้อน โดยในสารตัวนำและกึ่งตัวนำทั่วไปจะมีทั้งอิเลกตรอนอิสระ (free electrons) ซึ่งมีประจุลบ และหลุม (holes) ซึ่งมีประจุบวก เมื่อวัสดุได้รับความร้อน พาหะในสารจะแพร่ตัวออกไปยังบริเวณที่มีอุณหภูมิต่ำกว่า การสะสมของพาหะเหล่านี้ทำให้เกินศักย์ไฟฟ้าขึ้น

\begin{figure}
  \centering
  \begin{tikzpicture}[>=latex]
    \node [draw, fill=LightSkyBlue, minimum height=4cm, minimum width=1.5cm](n){};
    \node [draw, xshift=2.5cm, fill=Pink, minimum height=4cm, minimum width=1.5cm](p){};
    \draw [<->, ultra thick] (n.south west) ++ (180:1) --++ (90:4) node[midway, fill=white]{\Large\bfseries voltage};
    \draw [->] (n.north) ++ (-90:1) --++ (-90:2) node[midway, fill=LightSkyBlue]{$e^-$};
    \draw [->] (p.north) ++ (-90:1) --++ (-90:2) node[midway, fill=Pink]{$h^+$};
    \node [red, xshift=1.2cm, yshift=2.5cm]{\Large\bfseries Hot};
    \node [blue, xshift=1.2cm, yshift=-2.5cm]{\Large\bfseries Cold};
    \node at (n.south west) [xshift=-2mm] {\Large\bfseries -};
    \node at (n.north west) [xshift=-2mm] {\Large\bfseries +};
    \node at (p.south east) [xshift=2mm] {\Large\bfseries +};
    \node at (p.north east) [xshift=2mm] {\Large\bfseries -};
  \end{tikzpicture}
  \caption{หลักการของปรากฏการณ์เทอร์โมอิเล็กทริก}
\end{figure}

เมื่อนำไปต่อกับภาระภายนอกจะทำให้มีการไหลของกระแสไฟฟ้าเกินขึ้นได้

\begin{figure}[h]
  \centering
  \begin{tikzpicture}[>=latex]
    \node [draw, fill=LightSkyBlue, minimum height=4cm, minimum width=1.5cm](n){};
    \node [draw, xshift=2.5cm, fill=Pink, minimum height=4cm, minimum width=1.5cm](p){};
    % \draw [<->, ultra thick] (n.south west) ++ (180:1) --++ (90:4) node[midway, fill=white]{\Large\bfseries voltage};
    \draw [->] (n.north) ++ (-90:1) --++ (-90:2) node[midway, fill=LightSkyBlue]{$e^-$};
    \draw [->] (p.north) ++ (-90:1) --++ (-90:2) node[midway, fill=Pink]{$h^+$};
    \node [red, xshift=1.2cm, yshift=2.5cm]{\Large\bfseries Hot};
    \node [blue, xshift=1.2cm, yshift=-2.5cm]{\Large\bfseries Cold};

    \draw [line width=5pt] (n.north west) -- (p.north east);
    \draw [-*] (n.south west) --++ (180:1) --++ (-90:2) --++ (0:7.5);
    \draw [-*] (p.south east) --++ (0:2.5) node[yshift=-1cm]{\Large\bfseries $V$};
    \draw (p.south east) ++ (0:1) to [R,l_=$R_L$,i=$i$] ++(-90:2);
  \end{tikzpicture}
  \caption{การเชื่อมต่อเครื่องกำเนิดไฟฟ้าเทอร์โมอิเล็กทริก}
\end{figure}

สารทุกชนิดมีความสามารถในการสร้างศักย์ไฟฟ้าจากการแพร่ของพาหะประจุที่ต่างกัน โดยค่าความสามารถนี้เรียกว่า ค่าสัมประสิทธิ์ซีเบ็ก (Seebeck Coefficient) ซึ่งอธิบายความสามารถศักย์ไฟฟ้าที่เกิดจากอุณหภูมิที่แตกต่างได้ดังนี้

\begin{equation}
  \label{eq:seebeck coefficient}
  V = \int_{T_L}^{T_H} \left( S_p - S_n \right) dT = \int_{T_L}^{T_H} S_{pn} dT
\end{equation}

ซึ่งหากเราสมมติว่าค่าสัมประสิทธิ์นี้เป็นอิสระจากอุณหภูมิ จะสามารถเขียนสมการ \ref{eq:seebeck coefficient} ใหม่ได้ว่า

\begin{equation}
  \label{eq:seebeck coefficient2}
  V = S_{pn} \Delta T = S_{pn} \left( T_H - T_L \right)
\end{equation}

โดยค่าสัมประสิทธิ์สำหรับวัสดุทั่วไปที่มีสมบัติเป็นวัสดุเทอร์โมอิเล็กทริกได้มีดังนี้

\begin{table}[htbp]
\caption{\label{tab:org8ebcf41}
ค่าสัมประสิทธิ์ซีเบ็กของวัสดุต่างๆที่ 25 C}
\centering
\begin{tabular}{lr}
\hline
Material & \(S\), V / K \(\times 10^{-6}\)\\
\hline
Aluminum & -0.2\\
Constantan & -47\\
Copper & 3.5\\
Iron & 13.6\\
Platinum & -5.2\\
Germanium & 375\\
Silicon & -455\\
Bismuth Telluride & 200\\
\hline
\end{tabular}
\end{table}

อย่างไรก็ดี ประสิทธิภาพของเทอร์โมอิเล็กทริกจากวัสดุหนึ่งๆนั้นไม่ได้ขึ้นอยู่กับค่าสัมประสิทธิ์ซีเบ็กเพียงอย่างเดียว เนื่องจากลักษณะการทำงานและการต่อเชื่อมของเทอร์โมอิเล็กทริกกับวงจรไฟฟ้านั้นเป็นเหมือนแบตเตอรี่ชนิดหนึ่ง ซึ่งสามารถเขียนอธิบายเป็นวงจรได้ดังนี้

\begin{figure}[h]
  \centering
  \begin{tikzpicture}
    \draw (0,2) to [V_=$V_{OC}$] ++(90:-2) to [short, -*] ++ (0:2) to [short] ++ (0:2) to [R,l_=$R_L$] ++ (90:2) to [short, -*] ++ (180:2) to [R=$R_{TEG}$, -*] ++ (180:2);
    \node [yshift=1cm, xshift=2.5mm, draw, dashed, rounded corners=4mm, minimum width=3.5cm, minimum height=3.5cm](teg){};
    \node [below=of teg, yshift=1cm] {Thermoelectric Generator};
  \end{tikzpicture} 
  \caption{ภาพวงจรแสดงคุณสมบัติของเครื่องผลิตไฟฟ้าเทอร์โมอิเล็กทริก}
  \label{fig:thermoelectric circuit}
\end{figure}

จากรูปที่ \ref{fig:thermoelectric circuit} จะเห็นว่าเทอร์โมอิเล็กทริกเป็นเหมือนแหล่งศักย์ไฟฟ้า (\(V\)) และมีความต้านทานภายใน (\(R_{TEG}\)) 

\begin{gather}
  V_L = S_{pn}\Delta T - iR_{int} \\
  R_{int} = R_p + R_n
\end{gather}

นอกจากนี้ อีกวิธีที่จะเพิ่มปริมาณไฟฟ้าก็คือการต่อคู่เทอร์โมอิเล็กตริกแบบอนุกรมเพื่อเพิ่มแรงดันไฟฟ้า เหมือนอย่างเวลาต่อแบตเตอรี่ AA หรือ AAA หลายก้อนในอุปกรณ์ไฟฟ้าแบบพกพาทั้งหลาย ถ้าสมมุติว่าต่อเทอร์โมอิเล็กทริกทั้งหมด \(m\) คู่ จะได้สมการไฟฟ้าว่า

\begin{gather}
  V = m S_{pn} \Delta T \\
  R_{teg} = m R_{int} \\
  V_L = m S_{pn} \Delta T - i mR_{int}
\end{gather}

การที่จะสามารถดึงกำลังไฟฟ้าจากเทอร์โมอิเล็กทริกมาใช้ให้ได้มากที่สุดจึงจำเป็นจะต้องมีการปรับความต้านทานภาระ (Load resistance, \(R_L\)) ให้เหมาะสม เพื่อให้มีการสูญเสียไปกับความต้านทานภายในของเทอร์โมอิเล็กทริกให้น้อยที่สุด ซึ่งความต้านทานภาระที่เหมาะสมนี้สามารถหาได้จากสมการดังนี้

\begin{gather*}
  P_L = iV_L = i m S_{pn} \Delta T - i^2 m R_{int} \\
  \frac{d P_L}{d i } = 0 = m(S_{pn} \Delta T - 2 i R_{int}) \\
  i_{max P} = \dfrac{S_{pn} \Delta T}{2 R_{int}} \\
  i = \dfrac{V}{R} = \dfrac{ m S_{pn} \Delta T }{ m R_{int} + R_L } \\
  R_L = m R_{int}
\end{gather*}

หมายความว่า ความต้านทานภาระควรจะเท่ากับความต้านทานภายใน ซึ่งนี่เรียกว่า load matching ซึ่งเป็นวิธีการที่ใช้ได้กับการผลิตไฟฟ้าด้วยกระบวนการอื่นๆได้เช่นกัน

\subsection{ปรากฏการณ์เพลเทียร์ (Peltier Effect)}
\label{sec:org9c9337e}

เป็นปรากฏการณ์ที่ "ตรงกันข้าม" กับปรากฏการณ์ซีเบ็ก ในกรณีของปรากฏการณ์ซีเบ็กนั้น ผลต่างของอุณหภูมิสร้างให้เกิดความต่างศักย์และกระแสไฟฟ้า ส่วนปรากฏการณ์เพลเทียร์เป็นการสร้างผลต่างของอุณหภูมิเมื่อมีกระแสไฟฟ้าไหลผ่าน เปรียบเทียบได้กับกรณีของปรากฏการณ์แม่เหล็กไฟฟ้าในมอเตอร์ ซึ่งเมื่อใส่กระแสไฟฟ้าเข้าไปในตัวนำซึ่งอยู่ในสนามแม่เหล็กจะทำให้เกิดการหมุน ในทางตรงกันข้าม ถ้านำตัวนำไปหมุนภายในสนามแม่เหล็กก็จะทำให้เกิดกระแสไฟฟ้าเหนี่ยวนำขึ้นเช่นกัน

\begin{figure}[h]
  \centering
  \begin{tikzpicture}[>=latex]
    \node [draw, fill=LightSkyBlue, minimum height=3cm, minimum width=1.2cm](n){};
    \node [draw, xshift=2.5cm, fill=Pink, minimum height=3cm, minimum width=1.2cm](p){};
    % \draw [<->, ultra thick] (n.south west) ++ (180:1) --++ (90:4) node[midway, fill=white]{\Large\bfseries voltage};
    \node at (n.north) [below, fill=none]{$n$};
    \node at (p.north) [below, fill=none]{$p$};
    \node [blue, xshift=1.2cm, yshift=2cm]{\Large\bfseries Cold};
    \node [red, xshift=1.2cm, yshift=-1.7cm]{\Large\bfseries Hot};

    \draw [line width=5pt] (n.north west) -- (p.north east);
    \draw [short] (n.south west) --++ (180:1) --++ (-90:2) --++ (0:5.7);
    \draw [short] (p.south east) --++ (0:1); 
    \draw (p.south east) ++ (1,-2) to [battery1,i=$i$] ++(90:2);
  \end{tikzpicture}
  \caption{วงจรแสดงการเกิดปรากฏการณ์เพลเทียร์}
\end{figure}

ประโยชน์ของปรากฏการณ์นี้สามารถนำไปประยุกต์ใช้ในการทำความเย็น โดยตัวทำความเย็นที่อาศัยหลักการนี้เรียกว่าตัวทำความเย็นเพลเทียร์ (Peltier cooler) โดยอัตราการกำจัดความร้อนสามารถคำนวณได้จาก

$$ Q_{peltier} = m S_{pn} T_H i $$

ซึ่งตัวทำความเย็นนี้มีจุดเด่นเช่นเดียวกับตัวผลิตไฟฟ้าเทอร์โมอิเลกตริก นั่นคือไม่มีชิ้นส่วนที่เคลื่อนไหว จึงทำให้มีอัตราการสึกหรอน้อยกว่าระบบทำความเย็นแบบใช้สารทำความเย็นทั่วไป ลดความซับซ้อนของระบบทำความเย็น รวมถึงลดค่าซ่อมแซมและดูแลรักษาได้ แม้ปัจจุบันประสิทธิภาพจะยังไม่ดีเท่ากับระบบทำความเย็นแบบทั่วไป และมีราคาสูงเมื่อเทียบกับอัตราการกำจัดความร้อน แต่ก็ได้มีการนำมาใช้ในกรณีที่มีพื้นที่การติดตั้งจำกัด เช่นระบบทำความเย็นในหน่วยประมวลผล (processor) ของคอมพิวเตอร์

\subsection{ปรากฏการณ์ทอมสัน (Thomson Effect)}
\label{sec:orgb13f282}

ดังที่ได้กล่าวมาแล้วในส่วนของปรากฏการณ์เทอร์โมอิเลกทริก ค่าสัมประสิทธิ์ซีเบ็กของแต่ละวัสดุนั้นมักจะแปรผันกับอุณหภูมิ ดังนั้นในกรณีที่วัสดุมีอุณหภูมิที่ไม่สม่ำเสมอ ค่าสัมประสิทธิ์ซีเบ็กก็อาจจะไม่สม่ำเสมอได้เช่นกัน และเมื่อมีกระแสไฟฟ้าไหลผ่านวัสดุนี้ก็จะทำให้มีการเกิดปรากฏการณ์เพลเทียร์เกิดขึ้นได้ ปรากฏการณ์นี้เรียกว่า`ปรากฏการณ์ทอมสัน' ตั้งตามชื่อของลอร์ดเคลวิน (ชื่อจริง William Thomson) ซึ่งได้ทำนายการเกิดปรากฏการณ์นี้ในตัวนำที่มีอุณหภูมิไม่สม่ำเสมอดังที่ได้กล่าวมาแล้วในส่วนของปรากฏการณ์เทอร์โมอิเลกทริก ค่าสัมประสิทธิ์ซีเบ็กของแต่ละวัสดุนั้นมักจะแปรผันกับอุณหภูมิ ดังนั้นในกรณีที่วัสดุมีอุณหภูมิที่ไม่สม่ำเสมอ ค่าสัมประสิทธิ์ซีเบ็กก็อาจจะไม่สม่ำเสมอได้เช่นกัน และเมื่อมีกระแสไฟฟ้าไหลผ่านวัสดุนี้ก็จะทำให้มีการเกิดปรากฏการณ์เพลเทียร์เกิดขึ้นได้ ปรากฏการณ์นี้เรียกว่า`ปรากฏการณ์ทอมสัน' ตั้งตามชื่อของลอร์ดเคลวิน (ชื่อจริง William Thomson) ซึ่งได้ทำนายการเกิดปรากฏการณ์นี้ในตัวนำที่มีอุณหภูมิไม่สม่ำเสมอและทำการทดลองจนสามารถพิสูจน์ได้จริง 

ในกรณีที่มีความหนาแน่นกระแสไฟฟ้า \(J\) ไหลผ่านตัวนำที่มีค่าสัมประสิทธิ์ทอมสัน \(\mathcal{K}\) อัตราการเกิดความร้อนจะมีค่าเท่ากับ

$$ q_{thomson} = - \mathcal{K} J \cdot \nabla T $$

สังเกตว่าในสมการนี้ กำลังความร้อนที่เกิดขึ้นมืหน่วยเป็น W/m\(^3\) เนื่องจากคุณสมบัติของตัวนำไม่สม่ำเสมอ กำลังความร้อนจึงไม่คงที่และต้องอาศัยการอินทิเกรตเพื่อหาค่าบนพื้นที่หรือปริมาตร

\subsection{หลักการทำงานของเทอร์โมอิเลกทริก}
\label{sec:orgb5d7181}

ในระหว่างการทำงานจริงมักมีปรากฏการณ์เทอร์โมอิเลกทริกสองอย่างขึ้นไปเกิดขึ้นพร้อมๆกัน ดังนั้นจึงมีความจำเป็นที่จะต้องทำความเข้าใจความสัมพันธ์ของปรากฏการณ์ต่างๆและผลที่เกิดขึ้นกับเทอร์โมอิเลกทริก อย่างไรก็ดี สำหรับในตำราเล่มนี้ จะขอกล่าวถึงความสัมพันธ์เมื่อเทอร์โมอิเลกทริกทำงานที่สถานะคงที่ (steady state) ซึ่งหมายถึงอุณหภูมิที่จุดต่างๆคงที่ ในที่นี้เราจะพิจารณาที่ด้านร้อนของเทอร์โมอิเลกทริกซึ่งมีการถ่ายเทความร้อนเกิดขึ้นดังต่อไปนี้

\begin{enumerate}
\item ความร้อนจากแหล่งความร้อนเข้าสู่ด้านร้อน $Q_{in}$
\item ความร้อนจากปรากฏการณ์การเกิดความร้อนของจูล $Q_{joule}$

  $$ Q_{joule} = i^2 R $$

\item ความร้อนออกจากด้านร้อนไปสู่ด้านเย็นด้วยการนำความร้อน $Q_{cold}$

  $$ Q_{cold} = K \Delta T $$

\item ความร้อนออกจากด้านร้อนด้วยปรากฏการณ์เพลเทียร์ $Q_{peltier}$

  $$ Q_{peltier} = S_{pn} T_H i $$
\end{enumerate}

ที่สถานะคงที่ อัตราการได้รับความร้อนและสูญเสียความร้อนเท่ากัน ซึ่งอัตราการได้รับความร้อน (\(Q_{in}\)) มาจาก

$$ Q_{in} + Q_{joule} = Q_{cold} + Q_{peltier}$$

\begin{equation}
\begin{aligned}
  Q_{in} &=  Q_{cold} + Q_{peltier} - Q_{joule}  \\
  &=  m S_{pn} T_H i +  K\Delta T -  \dfrac{i^2 R_{teg}}{2}
\end{aligned}
\end{equation}

กำลังไฟฟ้าที่ผลิตได้ผ่านตัวต้านทานเท่ากับ

\begin{equation}
  P_{out} = i^2 R_L
\end{equation}

ซึ่งเราสามารถเอามาเขียนเป็นสมการประสิทธิภาพความร้อนของ TEG เท่ากับ

\begin{equation}
  \label{eq: TEG thermal eff}
  \begin{aligned}
    \eta &= \frac{P_{out}}{Q_{in}} \\
    &= \frac{i^2 R_L}{ m S_{pn} T_H i + K \Delta T - \dfrac{ i^2 R_{teg}}{2}}
  \end{aligned} 
\end{equation}

เมื่อแทนค่า \(Z = \dfrac{m^2 S_{pn}^2}{K_{teg} R_{teg}}\) เข้าในสมการ \ref\{eq: TEG thermal
eff\}

\begin{equation}
  \eta = \dfrac{ \Delta T }{ 2 T_H + \dfrac{2}{Z} - \dfrac{ \Delta T }{ 2 } }
\end{equation}

จากสมการข้างต้น ที่อุณหภูมิ \(T_H\) และ \(T_L\) ใดๆ ประสิทธิภาพของ TEG จะสูงสุดเมื่อมีค่า \(Z\) สูง ซึ่งแปลว่าวัสดุจะต้องมีค่าสัมประสิทธ์ซีเบ็กสูง นำความร้อนได้ไม่ดี และมีความต้านทานไฟฟ้าต่ำ ซึ่งคุณสมบัติสองอย่างหลังนี้หาได้ยาก เพราะวัสดุที่เป็นตัวนำไฟฟ้าที่ดี ก็มักจะนำความร้อนได้ดีเช่นกัน ส่วนวัสดุที่เป็นฉนวนไฟฟ้า ก็มักจะเป็นฉนวนความร้อนด้วย

ประสิทธิภาพของเครื่องยนต์ความร้อนส่วนใหญ่ (นอกจากเครื่องยนต์สันดาปภายใน) มักจะเปรียบเทียบประสิทธิภาพเป็นสัดส่วนเทียบกับประสิทธิภาพคาร์โนต์ซึ่งเป็นประสิทธิภาพสูงสุดในทางทฤษฎีของเครื่องยนต์ความร้อนใดๆ

\begin{figure}[h]
  \centering
  \begin{tikzpicture}
    \begin{axis} [
      width=\textwidth,
      height=.7\textwidth,
      legend style={at={(0.1,0.9)},
        legend cell align={left},
        anchor=north west,
        fill=none},
      % xtick=data,
      xmin=0,xmax=400,
      ymin=0,ymax=0.6,
      domain=0:400,
      ytick distance=0.1,
      xlabel={ส่วนต่างอุณหภูมิ $\Delta T$},
      ylabel={ประสิทธิภาพความร้อน \%},
      % cycle list/Paired,
      ]
      \addplot {\x / (50 + \x + 273)};
      \addplot {\x / (2*(\x + 50 + 273) + 2*(\x + 273 + 50)/2 - \x/2) };
      \addplot {\x / (2*(\x + 50 + 273) + 2*(\x + 273 + 50)/1 - \x/2) };
      \addplot {\x / (2*(\x + 50 + 273) + 2*(\x +  273 + 50)/0.5 - \x/2) };
      \legend{Carnot, ZT = 2, ZT = 1, ZT = 0.5};
    \end{axis}
  \end{tikzpicture}
  \caption{ประสิทธิภาพความร้อนของ TEG เทียบกับประสิทธิภาพคาร์โนต์}
  \label{fig:teg vs carnot efficiency}
\end{figure}

\subsection{ตัวอย่าง}
\label{sec:orgfd6cf17}
เทอร์โมอิเลกทริกทำมาจาก PbTe-Bi\(_{\text{2}}\)Te\(_{\text{3}}\) ซึ่งมีคุณสมบัติดังต่อไปนี้

\begin{center}
\begin{tabular}{lrr}
\hline
Properties & P-type & N-type\\
\hline
Seebeck coefficient \texttimes{} 10\(^{\text{-6}}\) & 300 & -100\\
Electrical resistivity \texttimes{} 10\(^{\text{-6}}\) & 9 & 10\\
Thermal conductiviity & 1.2 & 1.4\\
\hline
\end{tabular}
\end{center}

ขาจากวัสดุทั้งสองชนิดมีพิ้นที่หน้าตัด (16 mm\(^2\))และความยาว (4 mm) เท่ากัน ที่สภาวะคงที่อุณหภูมิด้านร้อนเท่ากับ 200 C และด้านเย็นเท่ากับ 50 C จงคำนวณหา

\begin{enumerate}
\item ค่า \(Z\) ของเทอร์โมอิเลกทริกนี้
\item กำลังสูงสุดที่เทอร์โมอิเลกทริกนี้ผลิตได้
\item ประสิทธิภาพของเทอร์โมอิเลกทริกนี้
\end{enumerate}

\subsection{ต้นทุนของพลังงานจากเทอร์โมอิเลกทริก}
\label{sec:org66672ad}

ต้นทุนวัสดุที่ใช้ทำเทอร์โมอิเลกทริกในปัจจุบัน

\begin{table}[htbp]
\caption{\label{tab:orgf38be2f}
ค่าวัสดุเทอร์โมอิเลกทริก}
\centering
\begin{tabular}{lrrll}
\hline
Material Family & Max ZT & Temp (°C) & Efficiency & Average Material Cost (\$/kg)\\
\hline
Cobal Oxide & 1.4 & 727 & 12\% & \$345\\
Cobalt Oxide & 1.4 & 727 & 12\% & \$345\\
Clathrate & 1.4 & 727 & 12\% & \$5,310\\
SiGe & 0.86 & 727 & 9\% & \$6,033\\
Chalcogenide & 2.27 & 727 & 16\% & \$730\\
Half-Heusler & 1.42 & 427 & 17\% & \$1,988\\
Skutterudite & 1.5 & 427 & 18\% & \$562\\
Silicide & 0.93 & 727 & 9\% & \$151\\
\hline
\end{tabular}
\end{table}

\subsection{ตัวอย่าง}
\label{sec:orgb51cf83}
การเปรียบเทียบต้นทุนการผลิตไฟฟ้าจากเทอร์โมอิเกลกทริกด้วยอุณหภูมิขนาดกลาง

กรณีเปรียบเทียบ 3 แบบ: น้ำมันเตาเป็นเชื้อเพลิง ความร้อนเหลือทิ้งจากอุตสาหกรรม หรือซื้อไฟฟ้าจากการไฟฟ้าฯ

สมมติฐานที่ใช้ในการวิเคราะห์ 

\begin{enumerate}
\item การผลิตไฟฟ้าขนาด 1 MW โดยสิ่งก่อสร้างและอุปกรณ์ทั้งหมดมีอายุการใช้งาน 10 ปี
\item ต้นทุนคงที่จากอุปกรณ์เทอร์โมอิเลกทริก อินเวอร์เตอร์ ค่าที่ดิน และค่าติดตั้ง
\item ต้นทุนแปรผันนับจากค่าซ่อมแซมและค่าเชื้อเพลิง(ถ้ามี)
\item ค่าอินเวอร์เตอร์ 22 บาทต่อวัตต์ ค่าเทอร์โทอิเลกทริกอุณหภูมิสูง 175 บาทต่อวัตต์ ค่าเทอร์โมอิเลกทริกอุณหภูมิกลาง 525 บาทต่อวัตต์
\item ค่าติดตั้ง 10\% ของค่าอุปกรณ์ (TEG + Inverter)
\item ค่าซ่อมแซม 1\% ของค่าอุปกรณ์ต่อปี
\end{enumerate}

ก่อนอื่น เราสามารถคำนวณค่าอุปกรณ์ที่ต้องใช้ในการแปลงไฟฟ้า ซึ่งประกอบด้วยค่า TEG และ inverter 

เปรียบเทียบต้นทุนระหว่างกรณีที่ 1, 2, และ 3 ได้เป็นตารางดังนี้

\begin{center}
\begin{tabular}{lrr}
\hline
Costs (million THB) & Fuel & Waste\\
\hline
TEGs & 175 & 525\\
Inverters & 22 & 22\\
Land & 1 & 1\\
Installation & 20 & 55\\
Maintenance (per year) & 2 & 5.5\\
Fuel (per year) & 191 & 0\\
\hline
\end{tabular}
\end{center}

และยังสามารถแสดงกระแสเงินสดเปรียบเทียบระหว่างกรณีได้ดังนี้

\pgfplotstableread[row sep=\\,col sep=&]{
  year     &  Fuel   & Waste \\
  0        & -218    & -603  \\
  1        & -191    & -5.5  \\
  2        & -191    & -5.5  \\
  3        & -191    & -5.5  \\
  4        & -191    & -5.5  \\
  5        & -191    & -5.5  \\
  6        & -191    & -5.5  \\
  7        & -191    & -5.5  \\
  8        & -191    & -5.5  \\
  9        & -191    & -5.5  \\
  10       & -191    & -5.5  \\
}\mydata

\begin{tikzpicture}
  \begin{axis}[
    ybar,
    bar width=.2cm,
    width=\textwidth,
    height=.5\textwidth,
    legend style={at={(0.75,0.25)},
      anchor=south east,legend columns=-1,
      fill=none},
    xtick=data,
    xlabel={Year},
    nodes near coords,
    nodes near coords align={vertical},
    ymin=-650,ymax=0,
    ylabel={Million THB},
    ]
    \addplot table[x=year,y=Fuel]{\mydata};
    \addplot table[x=year,y=Waste]{\mydata};
    \legend{Fuel, Waste};
  \end{axis}
\end{tikzpicture}

ในขณะเดียวกัน ค่าไฟฟ้าที่ซื้อจากการไฟฟ้าฯสามารถสมมติว่าเป็นค่าคงที่ในแต่ละปี ซึ่งหากเปรียบเทียบกับการลงทุนในระบบ TEG ทั้งสองแบบแล้ว จะสามารถหาผลต่างของกระแสเงินสดเพื่อจะนำไปใช้หาโครงการที่มีมูลค่าปัจจุบันสุทธิ (NPV) สูงสุดได้ดังนี้

\begin{center}
\begin{tabular}{rrrrrr}
\hline
Year & Base & Fuel & Waste & Fuel - Base & Waste - Base\\
\hline
0 & 0 & -218 & -603 & -218 & -603\\
1 & -39.4 & -191 & -5.5 & -151.6 & 33.9\\
2 & -39.4 & -191 & -5.5 & -151.6 & 33.9\\
3 & -39.4 & -191 & -5.5 & -151.6 & 33.9\\
4 & -39.4 & -191 & -5.5 & -151.6 & 33.9\\
5 & -39.4 & -191 & -5.5 & -151.6 & 33.9\\
6 & -39.4 & -191 & -5.5 & -151.6 & 33.9\\
7 & -39.4 & -191 & -5.5 & -151.6 & 33.9\\
8 & -39.4 & -191 & -5.5 & -151.6 & 33.9\\
9 & -39.4 & -191 & -5.5 & -151.6 & 33.9\\
10 & -39.4 & -191 & -5.5 & -151.6 & 33.9\\
\hline
 &  &  & NPV & -\$1,390 & -\$340\\
\hline
\end{tabular}
\end{center}

จากผลการวิเคราะห์กระแสเงินสดจะเห็นได้ว่าโครงการสร้างโรงไฟฟ้า TEG ทั้งสองแบบยังมีมูลค่าปัจจุบันสุทธิเป็นลบ หมายความว่าโครงการทั้งสองยังมีผลตอบแทนที่ยังไม่น่าพอใจเมื่อเปรียบเทียบกับใช้กระแสไฟฟ้าจากการไฟฟ้าฯ

\section{เซลล์เชื้อเพลิง (Fuel Cells)}
\label{sec:org51996ac}

เซลล์เชื้อเพลิงเป็นอุปกรณ์ที่อาศัยกระบวนการเปลี่ยนแปลงพลังงานจากพลังงงานเคมีไปเป็นพลังงานไฟฟ้าโดยตรง ซึ่งแตกต่างจากการใช้เครื่องยนต์ในการปั่นไฟซึ่งเปลี่ยนพลังงานเคมีไปเป็นพลังงานความร้อนไปเป็นพลังงานกลแล้วจึงเป็นพลังงานไฟฟ้าในที่สุด เนื่องจากเซลล์เชื้อเพลิงมีการเปลี่ยนแปลงพลังงานเพียงขั้นตอนเดียว และยังไม่มีขั้นตอนการเปลี่ยนแปลงพลังงานความร้อน จึงทำให้สามารถทำให้กระบวนการมีประสิทธิภาพสูงกว่าวิธีเปลี่ยนแปลงพลังงานเคมีในรูปแบบอื่น



จุดเด่นของเซลล์เชื้อเพลิงคือสามารถนำการแลกเปลี่ยนอิเลกตรอนที่เกิดขึ้นในปฏิกิริยาการสันดาปมาใช้ได้โดยตรง ซึ่งปฏิกิริยาที่เกิดขึ้นในเซลล์เชื้อเพลิงนี้เรียกว่า \textbf{ปฏิกิริยาไฟฟ้าเคมี (electrochemical reactions)} ซึ่งเป็นหลักการเดียวกันกับแบตเตอรี่ ข้อแตกต่างของแบตเตอรี่คือสารเคมีหรือเชื้อเพลิงทั้งหมดจะถูกบรรจุอยู่ในภายในตัวแบตเตอรี่ ในขณะที่เชื้อเพลิงของเซลล์เชื้อเพลิงถูกเก็บไว้แยกกัน และถูกดึงเข้ามาใช้เมื่อเกิดปฏิกิริยาขึ้นเท่านั้น

\subsection{ส่วนประกอบของเซลล์เชื้อเพลิง}
\label{sec:org982b3ba}

\subsection{ปฏิกิริยาในเซลล์เชื้อเพลิง}
\label{sec:orgf46d7fa}

อันที่จริงแล้ว ปฎิกิริยาที่เกิดขึ้นในเซลล์เชื้อเพลิงก็คือปฏิกิริยาการสันดาป แต่เนื่องจากเซลล์เชื้อเพลิงเป็นอุปกรณ์เคมีไฟฟ้า เราจึงควรทำความเข้าใจกับปริมาณของอิเลกตรอนที่มีการแลกเปลี่ยนระหว่างการเกิดปฏิกิริยาขึ้น ยกตัวอย่างเช่น

$$ H_2 + \frac{1}{2} O_2 \rightarrow H_2O $$

ในปฏิกิริยานี้ มีการแลกเปลี่ยนอิเลกตรอนระหว่างไฮโดรเจนกับออกซิเจน โดยที่ไฮโดรเจนเป็นผู้ให้ ส่วนออกซิเจนเป็นผู้รับ ซึ่งปฏิกิริยาเคมีที่มีการแลกเปลี่ยนอิเลกตรอน เรียกว่าปฏิกิริยารีดอกซ์ (redox reaction) ซึ่งมาจากการรวมกันของปฏิกิริยารีดักชัน (reduction reaction) และออกซิเดชัน (oxidation reaction) ซึ่งปฏิกิริยาข้างต้นสามารถแบ่งออกเป็นปฏิกิริยารีดักชันและออกซิเดชันได้ดังนี้

\paragraph{ปฏิกิริยารีดักชัน}

$$ 2H^+ + 2e^- + O_2 \rightarrow H_2O $$

\paragraph{ปฏิกิริยาออกซิเดชัน}

$$ H_2 \rightarrow 2H^+ + 2e^- $$

ในปฏิกิริยารีดักชัน สารจะมีการรับอิเลกตรอน (จาก \(H^+\) ซึ่งมีเลขประจุเป็น +1 ไปเป็น \(H_2O\) ซึ่งไฮโดรเจนมีประจุเป็น 0) ส่วนในปฏิกิริยาออกซิเดชัน สารจะมีการปล่อยอิเลกตรอน (จาก \(H_2\) ซึ่งมีประจุเป็น 0 เป็น \(H^+\) ซึ่งมีประจุเป็น +1)

\subsection{พลังงานที่ได้จากเซลล์เชื้อเพลิง}
\label{sec:orgf9bc12a}

พลังงานตั้งต้นของเซลล์เชื้อเพลิงมาจากพลังงานเคมีของสารตั้งต้น แล้วพลังงานเคมีคืออะไร พลังงานเคมีคือพลังงานที่ถูกเก็บไว้ในพันธะระหว่างอะตอมในโมเลกุลใดๆ และจะมีการเปลี่ยนแปลงเมื่อเกิดปฏิกิริยาสร้างผลิตภัณฑ์ใหม่ขึ้น ซึ่งพลังงานในพันธะเคมีเหล่านี้สามารถวัดได้โดยใช้ enthalpy of formation (\(\Delta H_f\)) ซึ่งพลังงานงานที่จะสามารถแปลงเป็นพลังงานไฟฟ้าได้มาจากพลังงานเคมีที่ได้รับการปลดปล่อยจากปฏิกิริยารีด็อกซ์ (\(\Delta H\))

$$ \Delta H = \sum (\Delta H)_{products} - \sum (\Delta H)_{reactants} $$

ค่า enthalpy of formation ของสารทั่วไปสามารถหาได้จากตาราง

\begin{equation*}
  C + O_2 \rightarrow CO_2
\end{equation*}

\begin{align*}
  \Delta H &= \sum (\Delta H)_{products} - \sum (\Delta H)_{reactants} \\
           &= \Delta H_{CO_2} - \Delta H_C - \Delta H_{O_2} \\
           &= -394 \times 10^3 - 0 - 0 \\
           &= -394 \times 10^3 \text{ J/mol } CO_2
\end{align*}

ในตัวอย่างนี้ พลังงานที่เปลี่ยนแปลงเป็นลบ แสดงว่าพลังงานของผลิตภัณฑ์น้อยกว่าของสารตั้งต้น หมายถึงมีการคายพลังงานออกมา ซึ่งเป็นปกติสำหรับปฏิกิริยาสันดาปทั่วไป เรียกได้อีกอย่างว่าปฏิกิริยาการคายพลังงาน (exothermic reaction)

แต่พลังงานที่คายออกมาไม่สามารถถูกแปลงเป็นพลังงานไฟฟ้าได้ทั้งหมด จะต้องมีการสูญเสียความร้อนเกิดขึ้นอย่างหลีกเลียงไม่ได้ ในกรณีที่ปฏิกิริยาเป็นแบบย้อนกลับได้ การสูญเสียพลังงานความร้อนเท่ากับ

\begin{equation}
  \text{Heat Loss} = \int T dS
\end{equation}

ที่สภาวะคงที่ การสูญเสียความร้อนจะกลายเป็น

\begin{equation}
  \text{Heat Loss} = T \Delta S
\end{equation}

หากเซลล์เชื้อเพลิงมีประสิทธิภาพ 100\% พลังงานเคมีที่เหลือจะสามารถแปลงไปเป็นพลังงานไฟฟ้าได้ทั้งหมด

\begin{equation}
  \label{eq:max electrical work}
  W_e = \Delta H - T \Delta S
\end{equation}

แต่หากปฏิกิริยาไม่ได้เกิดแบบย้อนกลับได้ พลังงานไฟฟ้าที่ได้จะน้อยกว่านี้

\subsubsection{พลังงานอิสระของกิบส์ (Gibbs Free Energy)}
\label{sec:orgc4a6bfc}

พลังงานอิสระของกิบส์ (GFE) เป็นฟังก์ชันสภาวะ (state function) ค่าสัมบูรณ์ของพลังงานอิสระของกิบศ์หาได้ยากและไม่ได้มีประโยชน์นัก ส่วนที่มีประโยชน์จริงๆคือผลต่างหรือพลังงานที่เปลี่ยนไประหว่างสารตั้งต้นกับผลิตภัณฑ์ ซึ่งใช้อธิบายว่าปฏิกิริยาหนึ่งๆสามารถเกิดขึ้นเองได้หรือไม่ หาได้จาก

\begin{equation}
  \label{eq:gfe definition}
  G = H - TS
\end{equation}

เมื่อทำการหาอนุพันธ์ของ GFE ในกระบวนการที่มีอุณหภูมิคงที่ (isothermal process)

\begin{equation}
  \label{eq:gfe derivative}
  dG = dH - TdS
\end{equation}

สำหรับความเปลี่ยนแปลงเล็กน้อยของเอนทาลปีและเอนโทรปี

\begin{equation}
  \label{eq:gfe changes}
  \Delta G = \Delta H - T \Delta S
\end{equation}

ซึ่งมีค่าเท่ากันกับพลังงานไฟฟ้าสูงสุดที่เซลล์เชื้อเพลิงสามารถผลิตได้ในสมการ \ref{eq:max electrical work} ซึ่งพลังงานอิสระของกิบส์ที่เปลี่ยนแปลงในปฏิกิริยาใดๆสามารถเขียนเป็นสมการได้ดังนี้

\begin{equation}
  \label{eq:gfe changes in reaction}
  \Delta G = \sum \Delta G_{products} - \sum \Delta G_{reactants}
\end{equation}

จากสมการ \ref{eq:gfe changes in reaction} หากพิจารณาปฏิกิริยาของสารที่เป็นแก๊สอุดมคติ จะสามารถเขียนความสัมพันธ์ทางอุณหพลศาสตร์ได้ดังนี้

\begin{gather}
  dU = TdS - PdV \\
  H = U + PV
\end{gather}

หาค่าอนุพันธ์ของ \(H\) ได้

\begin{align}
  dH &= dU + PdV + VdP \nonumber \\
     &= TdS - PdV + PdV + VdP \nonumber \\
     &= TdS + VdP
\end{align}

จัดรูปสมการใหม่จะได้ว่า

\begin{equation}
  VdP = dH - Tds = dG
\end{equation} 

หากพิจารณาสารตั้งต้น 1 mol จะได้ว่า

\begin{gather*}
  \label{eq:ideal gas equation}
  PV = R_u T \\
  V = \dfrac{R_u T}{P}
\end{gather*}

พิจารณาเซลล์เชื้อเพลิงที่สภาวะคงที่ จะได้ว่า \(T\) เป็นค่าคงที่

\begin{gather}
  \label{eq:gfe integral equation}
  \int_{G_0}^G dG = \int_{P_0}^P \dfrac{R_uT}{P}dP \\
  G - G_0 = R_u T \ln \dfrac{P}{P_0}
\end{gather}

โดยกำหนดให้ \(G_0\) คือพลังงานอิสระของกิบส์อ้างอิงที่อุณหภูมิ 25 C และความดัน 1 บรรยากาศ ดังนั้น เราสามารถเขียนสมการพลังงานอิสระของกิบส์เป็นฟังก์ชันของอุณหภูมิและความดันได้โดย

\begin{equation}
  \label{eq:gfe ideal gas}
  G = G_0 + R_u T \ln P
\end{equation}

ซึ่งพลังงานอิสระของกิบส์ที่เปลี่ยนไปในเซลล์เชื้อเพลิงสามารถอ้างอิงค่า \(H_0\) และ \(G_0\) ได้จากตาราง \ref{tab:h0 and g0} 

\subsubsection{พลังงานอิสระของกิบส์ที่เปลี่ยนแปลงในปฏิกิริยาเคมี}
\label{sec:org7a46ff4}

ในปฏิกิริยาเคมี พลังงานอิสระของกิบส์ที่เปลี่ยนไปเท่ากับส่วนต่างระหว่างพลังงานของผลิตภัณฑ์กับสารตั้งต้น ยกตัวอย่างเช่นในกรณีของปฏิกิริยา

\begin{equation*}
  aA + bB \rightarrow cC + dD
\end{equation*}

พลังงานอิสระของกิบส์ที่เปลี่ยนไปเท่ากับ

\begin{gather}
  \Delta G = G_{0C} + G_{0D} - G_{0A} - G_{0B} - R_u T \left( \ln P_C^c + \ln P_D^d - \ln P_A^a - \ln P_B^b \right) \nonumber \\
  \Delta G = \Delta G_0 + R_u T \ln \dfrac{P_C^c P_D^d}{P_A^a P_B^b} 
\end{gather}

\begin{table}[htbp]
\caption{\label{tab:org39b51e7}
ตารางแสดงค่าเอนทาลปีและพลังงานอิสระของกิบส์ที่สถานะอ้างอิง (1 บรรยากาศ 298 K)}
\centering
\begin{tabular}{lrr}
\hline
Compound or ion & H\(_{\text{0}}\) (\texttimes{}10\(^{\text{3}}\) J/mol) & G\(_{\text{0}}\) (\texttimes{}10\(^{\text{3}}\) J/mol)\\
\hline
CO & -110 & -137.5\\
CO\(_2\) & -394 & -395\\
CH\(_4\) & -74.9 & -50.8\\
Water & -286 & -237\\
Steam & -241 & -228\\
LiH & +128 & +105\\
NaCO\(_2\) & -1122 & -1042\\
CO\(_3^{-2}\) & -675 & -529\\
H\(^+\) & 0 & 0\\
Li\(^+\) & -277 & -293\\
OH\(^-\) & -230 & -157\\
CH\$\(_{\text{3}}\)\$OH (gas) & -201 & -162.6\\
\hline
\end{tabular}
\end{table}

ถ้าหากพลังงานเคมีทั้งหมดสามารถแปลงเป็นพลังงานไฟฟ้าได้ และมีอิเลกตรอน \(n\) ตัวถูกปล่อยออกมาต่อ 1 โมเลกุลของสารตั้งต้น เราจะสามารถเขียนสมการได้ว่า

\begin{equation}
  \label{eq:gfe to elec work}
  W_e= \Delta G = q E_g 
\end{equation}

โดยที่ \(W_e\) คือพลังงานไฟฟ้า \(q\) คือประจุไฟฟ้าที่มีการแลกเปลี่ยน และ \(E_g\) คือศักย์ไฟฟ้าที่เกิดขึ้น

\subsubsection{ศักย์ไฟฟ้าจากเซลล์เชื้อเพลิง}
\label{sec:org67981d6}

จากสมการ \ref{eq:gfe to elec work} ศักย์ไฟฟ้าที่เซลล์เชื้อเพลิงสามารถสร้างได้เท่ากับพลังานอิสระที่เปลี่ยนไปหารด้วยประจุที่มีการแลกเปลี่ยน ดังนั้นหากทุกๆโมเลกุลของสารตั้งต้นมีการแลกอิเลกตรอน \(n\) ตัว สมการแสดงศักย์ไฟฟ้าต่อ 1 molของสารตั้งต้นจะเป็น

\begin{equation}
  \label{eq:nernst equation}
  E_g = \frac{W_e}{-nF} = E_g^0 + \frac{R_u T}{nF} \ln \frac{P_A^a P_B^b}{P_C^c P_D^d}
\end{equation}

โดยที่ \(F\) คือค่าคงที่ของฟาราเดย์ซึ่งมีค่าเท่ากับประจุของอิเลกตรอนจำนวน 1 mol \(= 6.02 \times 10^{23} \times 1.6 \times 19^{-19} = 9.65 \times 10^4 \text{ C}\) สมการ \ref{eq:nernst equation} นี้ถูกตั้งชื่อตามผู้ค้นพบว่า \textbf{สมการของเนิร์นสท์ (Nernst Equation)}

\subsubsection{ประสิทธิภาพของเซลล์เชื้อเพลิง}
\label{sec:org96f667f}

ในทางทฤษฎี หากพลังงานอิสระของกิบส์จากปฏิกิริยาทั้งหมดถูกแปลงเป็นพลังงานไฟฟ้า ประสิทธิภาพของเซลล์เชื้อเพลิงจะมีค่าสูงที่สุด

\begin{equation}
  \label{eq:fc max eff}
  \eta_{\max} = \frac{W_{e,\max}}{\Delta H} = \frac{\Delta G}{\Delta H} = 1 - \frac{T \Delta S}{\Delta H}
\end{equation}

ในทางปฏิบัติแล้ว ปฏิกิริยาเคมีที่เกิดในเซลล์เชื้อเพลิงมักจะมีการสูญเสียความร้อนและพลังงานอื่นๆ ทำให้ศักย์ไฟฟ้าไม่สูงถึง \(E_g\) ที่คำนวณได้ด้วยสมการของเนิร์นสท์ ประสิทธิภาพของเซลล์เชื้อเพลิงจะเหลือ

\begin{equation}
  \label{eq:fc act eff}
  \eta = \frac{W_e}{\Delta H} = \frac{nFV_L}{\Delta H}
\end{equation}

\subsection{ตัวอย่าง: ประสิทธิภาพของเซลล์เชื้อเพลิงไฮโดรเจน}
\label{sec:org5e0489f}

เซลล์เชื้อเพลิงไฮโดรเจนได้รับไฮโดรเจนจากถังอัดความดันที่ 5 atm ในขณะที่ออกซิเจนได้มาจากอากาศที่ 1 atm ผลิตภัณฑ์ที่ได้ออกมาเป็นไอน้ำที่ 1 atm อุณหภูมิขณะที่เซลล์ทำงานอยู่ที่ 200\$\(^{\^{}}\)\$C คำนวณศักย์ไฟฟ้าที่เซลล์ผลิตได้และประสิทธิภาพของเซลล์เชื้อเพลิงนี้

จากสมการการสันดาปไฮโดเจนในเซลล์เชื้อเพลิง

$$ H_2 + \frac{1}{2} O_2 \rightarrow H_2O $$

\begin{itemize}
\item อุณหภูมิที่เซลล์ทำงาน = 200\(^{\circ}\) C = 200 + 273 = 473 K
\item เนื่องจากอากาศมีออกซิเจนอยู่ประมาณ 21$\backslash$% ความดันของออกซิเจนเข้าสู่เซลล์มีค่าเป็น 0.21 \texttimes{} 1 = 0.21 atm
\item มีการปล่อยและรับอิเลกตรอน 2 ตัวต่อ 1 โมเลกุลของน้ำ (\(n = 2\))
\end{itemize}

จากสมการที่ \ref{eq:nernst equation} เราสามารถแทนค่าเพื่อหาศักย์ไฟฟ้าได้ดังนี้

$$  E_g = \frac{W_e}{-nF} = -\frac{\Delta G_0}{nF} + \frac{R_u T}{nF} \ln \frac{P_{H_2} P_{O_2}^{1/2}}{P_{H_2O}} $$

จะสามารถแทนค่าได้โดยอ้างอิงปริมาณต่อ 1 mol \(H_2O\)

\begin{align*}
E_g &= - \frac{-241(10^3) - 0 - (0.5)0}{2 \times 9.65 \times 10^4} + \frac{8.314(473)}{2 \times 9.65 \times 10^4} \ln \frac{5^1 0.21^{0.5}}{1^1} \\
    &= \frac{231 \times 10^3}{2 \times 9.65 \times 10^4} \\
    &= 1.197 \text{ V}
\end{align*}

ประสิทธิภาพของเซลล์เชื้อเพลิงสามารถหาได้จากสมการ \ref{eq:fc max eff} 

\begin{align*}
  \eta &= \frac{\Delta G}{\Delta H} \\
       &= \frac{-231 \times 10^3}{-286 \times 10^3} \\
       &= 80.8\%
\end{align*}
\end{เฉลย}
\subsection{ชนิดของเซลล์เชื้อเพลิง}
\label{sec:orga876e27}

\subsubsection{Proton Exchange Membrane (PEM)}
\label{sec:org6d54feb}

\subsubsection{Direct Methanol}
\label{sec:org170c14b}

\subsubsection{Solid Oxide}
\label{sec:org629f7b3}

\subsection{การวิเคราะห์ต้นทุนของเซลล์เชื้อเพลิง}
\label{sec:org4e46498}

% historical fuel cell power cost

\begin{figure}[h]
  \centering
  \begin{tikzpicture}
    \pgfplotstableread[row sep=\\,col sep=&]{
      Year  & Cost   \\
      2006  & 124    \\
      2007  & 106    \\
      2008  &  81    \\
      2009  &  69    \\
      2010  &  59    \\
      2011  &  57    \\
      2012  &  55    \\
      2013  &  55    \\
      2014  &  55    \\
      2020  &  40    \\
      2050  &  30    \\
    }\mydata
    
    \centering
    \footnotesize
    \begin{axis}[
      ybar,
      nodes near coords,
      width=\textwidth,
      height=0.5\textwidth,
      bar width=0.5cm,
      xlabel={Year},
      xtick=data,
      ytick distance=20,
      symbolic x coords={2006,2007,2008,2009,2010,2011,2012,2013,2014,2020,2050},
      legend style={at={(1,1)},anchor=north east, text=Black},
      ]
      \addplot table[x=Year, y=Cost]{\mydata};
    \end{axis}
  \end{tikzpicture}
  \caption{Historical and projected transportation fuel cell system cost}
  \label{fig:projected fc cost}
\end{figure}

% cost breakdown of PEMFC (Y. Wang et al., Applied Energy 88 (2011), 981-1007)
\begin{figure}[h]
  \centering
  \begin{tikzpicture}
    \pgfplotstableread[row sep=\\,col sep=&]{
      Year  & MEA  & Bipolar & BOS & BOP \\
      2007  &  40  &     5   &   6 &  60 \\
      2008  &  34  &     5   &   5 &  45 \\
      2009  &  29  &     5   &   4 &  35 \\
      2010  &  18  &     2   &   3 &  25 \\
      2015  &  12  &     2   &   2 &  18 \\
    }\mydata
    
    \centering
    \footnotesize
    \begin{axis}[
      ybar stacked,
      % nodes near coords,
      width=\textwidth,
      height=0.5\textwidth,
      bar width=0.5cm,
      xlabel={Year},
      ylabel={\$/kW},
      xtick=data,
      reverse legend,
      ytick distance=20,
      symbolic x coords={2006,2007,2008,2009,2010,2015},
      legend style={at={(1,1)},anchor=north east, text=Black},
      ]
      \addplot table[x=Year, y=MEA]{\mydata};
      \addplot table[x=Year, y=Bipolar]{\mydata};
      \addplot table[x=Year, y=BOS]{\mydata};
      \addplot table[x=Year, y=BOP]{\mydata};
      \legend{MEA, Bipolar, BOS, BOP};
    \end{axis}
  \end{tikzpicture}
  \caption{Historical and projected transportation fuel cell system cost}
  \label{fig:projected fc cost}
\end{figure}

\section{พลังงานลม (Wind Power)}
\label{sec:org842f398}

พลังงานลมนับเป็นอีกพลังงานหนึ่งที่เกิดจากการไหลของอากาศ ดังนั้นการแปลงพลังงานลมเป็นพลังงานไฟฟ้าจึงเป็นการแปลงพลังงานกลไปเป็นพลังงานไฟฟ้า ซึ่งในบทนี้เราจะมากล่าวถึงหลักการ วิธี และประสิทธิภาพของการแปลงพลังงานลมด้วยเทคโนโลยีปัจจุบัน รวมถึงการประยุกต์ใช้เทคโนโลยีเหล่านี้ในการผลิตไฟฟ้าจากระดับเล็กไปจนถึงระดับใหญ่

\subsection{หลักการแปลงพลังงานลม}
\label{sec:orgf30a7a6}

พลังงานลมเป็นพลังงานจลน์ที่มีส่วนประกอบมาจากมวลของอากาศและความเร็วลม ซึ่งโดยทั่วไปแล้ว คำจำกัดความของพลังงานจลน์คือ

$$ E = \frac{1}{2} mv^2 $$

แต่เนื่องจากลมมีการเคลื่อนที่ต่อเนื่อง จึงสะดวกกว่าที่จะอธิบายถึงพลังงานลมในรูปของ\textbf{กำลังลม}แทนโดยใช้อัตราการไหลของมวลแทน

\begin{equation}
  \label{eq:wind power}
  \frac{dE}{dt} = P_w = {1 \over 2} \dot{m} v^2
\end{equation}

หากเราสมมติว่าลมมีความเร็วคงที่ จะสามารถคำนวณอัตราการไหลของมวลได้ว่า

\begin{equation}
  \label{eq:wind mass flowrate}
  \dot{m} = \rho A v
\end{equation}

เมื่อแทนสมการ \ref{eq:wind mass flowrate} ลงในสมการ \ref{eq:wind power} จะได้สมการแสดงกำลังของลมที่ความเร็ว \(v\) 

\begin{equation}
 \label{eq:wind power v}
 P_w = {1 \over 2} \dot{m} v^2 = {1 \over 2} \rho A v^3
\end{equation}

ถ้ามีการติดตั้งอุปกรณ์เพื่อดักและแปลงกำลังลมนี้เป็นกำลังไฟฟ้า ความเร็วลมขาออก \(v_o\) ต้องน้อยกว่าความเร็วลมขาเข้า \(v_i\) ดังนั้นความเร็วลมและอัตราการไหลของมวลผ่านอุปกรณ์เฉลี่ยคือ

\begin{gather}
  v_{avg} = \frac{v_i + v_o}{2} \\
  \dot{m} = \frac{\rho A}{2} \left( v_i + v_o \right)
  \label{eq:average wind speed through turbine}
\end{gather}

ดังนั้น ในทางทฤษฎีแล้วกำลังที่อุปกรณ์ดึงมาจากลมได้เท่ากับผลต่างของกำลังลมขาเข้ากับขาออก

\begin{align*}
  P_{output} &= P_{i} - P_{o} \\
              &= \frac{\dot{m}}{2} \left( v_i^2 - v_o^2 \right) \\
              &= \frac{\rho A}{4} \left( v_i + v_o \right)\left( v_i^2 - v_o^2 \right)
\end{align*}

ซึ่งเราสามารถใช้แคลคูลัสหาความเร็วลมขาออกซึ่งทำให้อุปกรณ์สามารถผลิตกำลังได้สูงสุด โดยการหาอนุพันธ์ของสมการกำลังแล้วตั้งให้เท่ากับศูนย์เพื่อแก้สมการ

\begin{gather}
 \frac{dP_{turbine}}{dk} = 0 = \frac{d}{dk} \left[ \frac{\rho A v_i^3}{4} \left( 1 + k \right) \left( 1 - k^2 \right) \right] \nonumber \\
    0 = \frac{d}{dk} \left[ 1 + k - k^2 - k^3 \right] \nonumber \\
    0 = 1 - 2k - 3k^2 \nonumber \\
    k = \frac{1}{3}, -1
\end{gather}

เนื่องจากลมขาออกไม่สามารถไหลย้อนกลับได้ (\(v_o\) เท่ากับ \(-v_i\) ไม่ได้) ดังนั้นคำตอบสมการเดียวที่เป็นไปได้คือ \(v_o = v_i /3\) ซึ่งทำให้อุปกรณ์ในอุดมคติสามารถเก็บกำลังลมได้

\begin{gather*}
    v_o = \frac{v_i}{3} \\
    P_{turbine, \max} = \frac{8}{27} \rho A v_i^3 = \frac{16}{27} P_{in} \\
    \eta_{\max} = \frac{16}{27} = 59.3\%
  \end{gather*}

ซึ่งค่าสูงสุดนี้เรียกว่า จำกัดของเบ็ทซ์ (Betz limit) อย่างไรก็ดี การวิเคราะห์แบบนี้มิได้มีการคำนึงถึงลักษณะทางอากาศพลศาสตร์ของอุปกรณ์ว่ามีผลต่อการไหลของอากาศอย่างไร

\subsection{อากาศพลศาสตร์ของกังหันลม}
\label{sec:org6ff954a}

อันที่จริงแล้ว การจะวิเคราะห์ประสิทธิภาพในการแปลงพลังงานของกังหันนั้นจำเป็นจะต้องพิจารณาการไหลของอากาศในขณะที่กังหันหมุนเพื่อพิจารณาแรงที่อากาศกระทำและกำลังที่เกิดขึ้น ซึ่งเราจะใช้หลักการอากาศพลศาสตร์เพื่อวิเคราะห์ประสิทธิภาพของกังหันลม

หากพิจารณาหลักการทางอากาศพลศาสตร์ กังหันลมที่มีใช้อยู่ในปัจจุบันสามารถแบ่งได้เป็น 2 ประเภทขึ้นอยู่กับแรงซึ่งขับเคลื่อนใบพัดในกังหัน

\begin{enumerate}
\item กังหันลมแรงต้าน
\item กังหันลมแรงยก
\end{enumerate}

ย้อนหลังไปถึงหลักอากาศพลศาสตร์ วัตถุใดๆที่ถูกลมกระทบจะเกิดแรงต้านและแรงยกขึ้น ซึ่งแรงทั้งสองสามารถเขียนเป็นสมการได้โดย

\begin{align}
  \label{eq:lift force}
  L = C_L \frac{1}{2} \rho A v^2 \\
  D = C_D \frac{1}{2} \rho A v^2
\end{align}

โดยที่ \(C_L\) และ \(C_D\) คือสัมประสิทธิ์แรงยกและสัมประสิทธิ์แรงต้าน ดังนั้น ในการสร้างกังหันลมจึงสามารถใช้แรงหนึ่งหรือทั้งสองในการขับดันและสร้างกำลัง โดยกำลังที่กังหันสามารถดึงออกมาได้ \(P_{turbine}\) เท่ากับผลคูณภายในของแรง \(\mathbf{F}\) และความเร็วของใบพัด \(\mathbf{u}\) 

\begin{equation}
  \label{eq:basic turbine power}
  P_{turbine} = \mathbf{F} \cdot \mathbf{u}
\end{equation}

ในกรณีของกังหันแบบแรงต้าน ทิศทางการไหลของลมจะไปในทิศทางเดียวกับแรงต้านเสมอ ดังนั้นสมการกำลังที่ผลิตได้จะมาจาก

\begin{gather}
  P = \mathbf{D} \cdot \mathbf{u} = \frac{1}{2} \rho A (v - u)^2 u \nonumber \\
  P = \frac{1}{2} \rho A C_D (uv^2 - 2vu^2 + u^3) \nonumber \\
  C_P = C_D \left( \lambda - 2\lambda^2 + \lambda^3 \right)
\end{gather}

โดยที่ \(\lambda = v / u\) เป็นอัตราส่วนของความเร็วลมต่อความเร็วกังหัน จะเห็นได้ว่าสัมประสิทธิ์กำลังที่ผลิตได้ \(C_p\) มีค่ามากที่สุดเมื่อ \(\lambda = 1/3\) เมื่อแทนค่าลงในสมการจะได้ว่า

\begin{equation}
  C_{P \max} = \dfrac{4}{27}C_D
  \label{eq:max power drag based}
\end{equation}

ซึ่งสำหรับกังหันที่มีสัมประสิทธิ์แรงต้านสูงอย่างเช่น \(C_D = 1.2\) จะได้ว่า \(C_P = 0.1778\)

ในกรณีของกังหันลมแรงยก ทิศทางการไหลของลมนั้นจะตั้งฉากกับความเร็วของใบพัดเสมอ ซึ่งทำให้ไม่มีข้อจำกัดเรื่องของความเร็วกังหันที่เร็วกว่าลม โดยที่รูปแสดงทิศทางของความเร็วและแรงที่เกิดขึ้นบนกังหันลมแรงยกสามารถแสดงได้ดังรูป

\begin{figure}[h]
  \centering
  \begin{tikzpicture}[>=latex]
    \draw [fill=White] (0,0) arc (280:90:0.15) node(A){}  arc (90:60:3) arc (71:100:2.9) -- cycle;
    \draw [->, thick] (A.north) --++ (90:2) node[midway, right]{$\mathbf{u}$};
    \draw [<-, thick] (A.south west) ++ (180:0.1) --++ (180:2.5) node[midway, above]{$\mathbf{v}$};
    \draw [<-, thick] (A.north west) --++ (142:3.1) node[midway, above right]{$\mathbf{w}$} node[at start, above, xshift=-1mm, yshift=2mm]{$\theta$};
    \draw [->, thick] (A.south east) ++ (-38:0.4)--++ (-38:1.8) node[midway, above right]{$\mathbf{D} = \frac{1}{2} C_D \rho A w^2$};
    \draw [->, thick] (A.north east) ++ (52:0.1)--++ (52:2.2) node[midway, below right]{$\mathbf{L} = \frac{1}{2} C_L \rho A w^2$};
  \end{tikzpicture}
  \caption{ทิศทางของความเร็วและแรงที่เกิดขึ้นบนกังหันลมแรงยก จะสังเกตได้ว่าในกรณีนี้ ความเร็วสัมพัทธ์ของลมเมื่อเทียบกับกังหันมีค่าเท่ากับ $w = \sqrt{ u^2 + v^2}$}
  \label{fig:lift-based turbine}
\end{figure}

ถ้าเรากำหนดให้ \(\gamma = \dfrac{C_D}{C_L}\) เป็นอัตราส่วนของแรงต้านต่อแรงยกที่เกิดขึ้น เราจะสามารถเขียนสมการแสดงกำลังที่กังหันลมแรงยกสร้างขึ้นได้ว่า

\begin{gather}
  P = (\mathbf{L} + \mathbf{D}) \cdot \mathbf{u} \nonumber \\
  P = \frac{1}{2}\rho A w^2 (C_L \frac{v}{w} u - C_D \frac{u}{w}u) \nonumber \\
  P = \frac{1}{2}\rho A \sqrt{u^2 + v^2} \left( C_L u v - C_D u^2 \right) \nonumber \\
   C_P = C_L \sqrt{1+\lambda^2} \left( \lambda - \gamma \lambda^2 \right) 
\end{gather}

สำหรับชิ้นส่วนภาคตัดขวางปีกอากาศยานทั่วไป \(\gamma =  0.01\) ที่ \(C_L = 0.6\)

\begin{figure}[h]
  \begin{tikzpicture}
    \begin{axis}[
      width=\textwidth,
      height=0.75\textwidth,
      legend style={at={(0.75,0.25)},
        anchor=south east,legend columns=-1,
        fill=none},
      xlabel={อัตราส่วนความเร็วกังหันต่อความเร็วลม, $\lambda$},
      xmin=0,xmax=100,
      ymin=0,ymax=1000,
      ylabel={สัมประสิทธิ์กำลัง, $C_P$},
      ]
      \addplot[domain=0:100, blue]{0.6*(1+\x^2)^0.5*(\x-0.01*\x^2)};
    \end{axis}
  \end{tikzpicture}
\end{figure}

จะเห็นได้ว่ากังหันลมแบบแรงยกนั้นมีประสิทธิภาพต่อพื้นที่ใบพัดสูงกว่ากังหันแบบแรงต้านหลายเท่าตัว จึงทำให้เป็นที่นิยมใช้ในอุตสาหกรรมผลิตไฟฟ้าพลังงานลมอย่างแพร่หลาย

\subsection{การออกแบบกังหันลมผลิตไฟฟ้า}
\label{sec:orga979388}

นอกจากเรื่องของการเลือกกังหันตามหลักการทำงานแล้ว ยังมีคุณลักษณะอื่นๆที่ผู้ใช้สามารถเลือกออกแบบกังหันลมได้ เช่น

\subsubsection{แนวแกนกังหัน\}}
\label{sec:org7efd506}

กังหันลมผลิตไฟฟ้ามีทั้งแบบที่มีแกนหมุนตามแนวนอนและแนวตั้ง ซึ่งแต่ละแบบมีข้อได้เปรียบเสียเแรียบอยู่ดังนี้

\begin{enumerate}
\item ค่าติดตั้งและซ่อมแซม กังหันแบบตั้งสามารถรับลมได้จากทุกทิศทาง และสามารถติดตั้งอุปกรณ์ปั่นไฟฟ้าไว้ใกล้กับพื้นได้ จึงสะดวกต่อการติดตั้งและซ่อมแซม ในขณะที่กังหันแบบแกนนอนจะต้องติดตั้งอุปกรณ์ทุกอย่างในแนวนอวเดียวกับกังหัน จึงมีค่าใช้จ่ายส่วนนี้ที่สูงกว่า
\item ประสิทธิภาพ เมื่อติดตั้งที่ความสูงที่สมควรและหันหน้าเข้าหาทิศทางลมแล้ว กังหันแบบแนวนอนจะมีประสิทธิภาพสูงกว่า
\end{enumerate}

\subsubsection{วัสดุผลิตกังหัน\}}
\label{sec:orgb51e0f1}

เนื่องจากกังหันต้องได้มีการหมุนอยู่ตลอดเวลา ภาระที่สำคัญที่ใบพัดจะได้รับคือแรงสู่ศูนย์กลางซึ่งขึ้นอยู่กับมวล ดังนั้นคุณสมบัติที่สำคัญสำหรับวัสดุที่จะนำมาใช้ออกแบบกังหันคือจะต้องมีอัตราส่วนความแข็งแรงต่อมวลสูง (high strength-to-mass ratio) ในอดีตวัสดุที่ใช้ในการผลิตกังหันลมได้แก่ ไม้เนื้อแข็ง (แข็งแรง น้ำหนักเบา แต่ไม่ทนทานต่อความชื้น)และโลหะเบาอย่างอลูมิเนียม (แข็งแรง เบา ขึ้นรูปง่าย แต่ไม่ทนทานต่อการล้า) ในปัจจุบันวัสดุที่ตอบโจทย์นี้ได้อย่างดีคือคาร์บอนไฟเบอร์เคลือบโพลีเมอร์ (CFRP) ซึ่งมีน้ำหนักเบาและความแข็งแรงสูง นอกจากนี้ยังสามารถขึ้นรูปเป็นรูปทรงที่ซับซ้อนได้ง่ายและมีความทนทานต่อการล้าได้ดี

\subsection{ต้นทุนการผลิตไฟฟ้าพลังงานลม}
\label{sec:org3764f43}

\begin{figure}[h]
  \centering
  \begin{tikzpicture}
    \pgfplotstableread[row sep=\\,col sep=&]{
      Cost Item  & Turbine    & BOS    & Financial \\
      Land-based & 1221       & 345    & 154       \\
      Offshore   & 1952       & 2277   & 1084      \\
    }\mydata
    
    \centering
    \begin{axis}[
      xbar stacked,
      width=0.9\textwidth,
      height=0.5\textwidth,
      bar width=1cm,
      xmin=0,
      xtick distance=1000,
      xlabel={Cost of Components (\$/kW)},
      ytick=data,
      symbolic y coords={Offshore, Land-based},
      % ytick=data,
      legend style={at={(1,1)},anchor=north east, text=Black},
      enlarge y limits=0.5,
      ]
      \addplot table[x=Turbine, y=Cost Item]{\mydata};
      \addplot table[x=BOS, y=Cost Item]{\mydata};
      \addplot table[x=Financial, y=Cost Item]{\mydata};
      \legend{Turbine,BOS,Financial};
    \end{axis}
  \end{tikzpicture}
  \caption{แผนภูมิเปรียบเทียบต้นทุนไฟฟ้าพลังงานลมจากกังหันบนบกและกังหันกลางทะเล}
  \label{fig:land vs short wind power}
\end{figure}

\begin{figure}[h]
  \begin{tikzpicture}
    \pgfplotstableread[row sep=\\,col sep=&]{
      Cost Item  & Turbine    & BOS    & Financial & OM       \\
      Land-based & 35         & 10     & 3         & 15       \\
      Offshore   & 51         & 60     & 29        & 39      \\
    }\mydata
    
    \centering
    \begin{axis}[
      xbar stacked,
      % nodes near coords,
      width=0.9\textwidth,
      height=0.5\textwidth,
      bar width=1cm,
      xmin=0,
      xlabel={Cost of Components (\$/MWh)},
      symbolic y coords={Offshore, Land-based},
      ytick=data,
      legend style={at={(1,1)},anchor=north east, text=Black},
      enlarge y limits=0.5,
      ]
      \addplot table[x=Turbine, y=Cost Item]{\mydata};
      \addplot table[x=BOS, y=Cost Item]{\mydata};
      \addplot table[x=Financial, y=Cost Item]{\mydata};
      \addplot table[x=OM, y=Cost Item]{\mydata};
      \legend{Turbine,BOS,Financial,O\&M};
    \end{axis}
  \end{tikzpicture}
  \caption{แผนภูมิเปรียบเทียบต้นทุนตลอดการใช้งานของโรงงานผลิตไฟฟ้าพลังงานลมแบบบนพิ้นดินกับแบบนอกชายฝั่ง}
\end{figure}
\section{พลังงานชีวภาพ (Biofuel)}
\label{sec:orgd9ed104}

\subsection{วัตถุดิบ (Feedstock)}
\label{sec:orgbaec029}
\subsection{เอทานอล}
\label{sec:org35e1ec2}

\subsection{โบโอดีเซล}
\label{sec:org5e9602b}

\subsection{แก็สชีวภาพ}
\label{sec:orgbf04eef}

\section{การกักเก็บพลังงาน (Energy Storage)}
\label{sec:org5223ae8}

ปัญหาหนึ่งของพลังงานแสงอาทิตย์และพลังงานทดแทนเช่นพลังงานลมหรือพลังงานคลื่นคือความไม่แน่นอนและไม่สามารถควบคุมได้ ซึ่งเป็นเกณฑ์วัดสำคัญของการสามารถพึ่งพาแหล่งพลังงานชนิดหนึ่งๆได้ ยกตัวอย่างเช่น ในกรณีของโรงไฟฟ้าพลังงานแก็สธรรมชาติ สามารถเปิดต่อเนื่องตลอดเวลาได้และสามารถเพิ่มหรือลดกำลังการผลิดได้ตามอุปสงค์อย่างไม่ยากเย็นนัก ในทางตรงกันข้าม พลังงานแสงอาทิตย์ไม่สามารถผลิตต่อเนื่องตลอดเวลาได้เนื่องจากช่วงเลากลางวันและกลางคืน นอกจากนี้ยังมีเรื่องของเมฆ ความชื้นในอากาศ ดังนั้น หากต้องการจะสร้างโรงไฟฟ้าพลังงานแสงอาทิตย์ (ไม่ว่าจะเป็นแบบ photovoltaics หรือ solar thermal หรือ แบบอื่นๆ) จำเป็นจะต้องสร้างเผื่อความไม่แน่นอนเหล่านี้ เช่นถ้ามีความต้องการพลังงานไฟฟ้า 10 MW อาจจะต้องสร้างโรงไฟฟ้าที่สามารถผลิตได้ 20 MW แล้วมีการกักเก็บส่วนที่เกินความต้องการไว้ใช้ในยามที่ไม่มีแสงอาทิตย์หรือพลังงานไม่เพียงพอต่อความต้องการผลิตต่อความต้องการผลิตต่อความต้องการผลิตต่อความต้องการผลิตต่อความต้องการผลิตต่อความต้องการผลิตต่อความต้องการผลิตต่อความต้องการผลิตต่อความต้องการผลิต 

ดังนั้น การจะลดผลกระทบจากความผันผวนและเพิ่มความสามารถในการควบคุมแหล่งพลังงาน โดยเฉพาะอย่างยิ่งแหล่งพลังงานทดแทนเช่นพลังงานแสงอาทิตย์ จำเป็นที่จะต้องมีอุปกรณ์กับเก็บพลังงานที่มีประสิทธิภาพเพื่อเก็บพลังงานส่วนเกินไว้ แล้วสามารถดึงพลังงานที่กเก็บไว้มาใช้ในช่วงที่มีความต้องการได้โดยไม่ต้องพึ่งพาแหล่งพลังงานโดยตรง

วิธีการกับเก็บพลังงานจากแสงอาทิตย์สามารถแบ่งได้เป็นหลายประเภท ซึ่งแต่ละประเภทก็มีจุดเด่นและจุดด้อยต่างกันไป พึงคำนึงไว้เสมอว่าไม่มีเทคโนโลยีใดที่จะเหมาะสมและดีกว่าเทคโนโลยีอื่นในทุกสถานการณ์ เราจึงควรทำความเข้าใจประเด็นต่างๆที่สำคัญเหล่านี้ไว้ เพื่อจะได้นำเทคโนโลยีเหล่านี้ไปประยุกต์ใช้ในสถานการณ์ต่างๆได้อย่างเหมาะสม

\subsection{บ่อกักเก็บพลังงานแสงอาทิตย์ (Solar Ponds)}
\label{sec:org441f245}

บ่อกักเก็บพลังงานแสงอาทิตย์ในที่นี้หมายถึงบ่อกับเก็บของเหลวซึ่งสามารถกับเก็บความร้อนจากพลังงานแสงอาทิตย์เพื่อนำไปใช้ประโยชน์ต่อไปได้ ในปัจจุบันบ่อกักเก็บพลังงานแสงอาทิตย์ส่วนมากใช้สารละลายเกลือคลอไรด์หรือซัลเฟตในน้ำ หลักการทำงานของบ่อดังกล่าวคือการแบ่งชั้นของสารละลายตามความความเข้มข้น โดยสารละลายที่มีความเข้มข้นมากจะตกอยู่ที่ชั้นล่างเนื่องจากมีความหนาแน่นสูง และสารละลายที่มีความเข้มข้นน้อยจะลอยอยู่ด้านบนเนื่องจากมีความหนาแน่นน้อย ซึ่งการแบ่งชั้นนี้จะป้องกันการหมุนเวียนของสารละลายเมื่อได้รับความร้อน ซึ่งในบ่อน้ำปกติเมื่อได้รับความร้อน จะมีการหมุนเวียนขึ้นเนื่องจากน้ำที่ร้อนกว่าจะมีการขยายตัว ทำให้ความหนาแน่นลดลงและลอยขึ้นสู่ด้านบน แต่ในบ่อน้ำที่มีการแบ่งชั้นของสานละลายนี้จะไม่มีการหมุนเวียนของสารละลาย ทำให้สามารถกักเก็บความร้อนไว้ได้

กระบวนการสร้างบ่อกักเก็บพลังงานแสงอาทิตย์มีอยู่ 2 วิธี

\subsubsection{บ่อกักเก็บแบบประดิษฐ์}
\label{sec:orgd31316a}

บ่อกักเก็บพลังงานแบบนี้สร้างโดยการเติมสารละลายที่มีความเข้มข้นจากสูงลงไปสู่ชั้นล่างแล้วลดลงต่ำลงเมื่อเพิ่มระดับน้ำขึ้นเรื่อยๆ จนเมื่อเติมเสร็จ บ่อก็จะสามารถกับเก็บพลังงานแสงอาทิตย์ไว้ได้

\subsubsection{บ่อกักเก็บแบบเกิดเอง}
\label{sec:org0fd23f5}

บ่อประเภทนี้อาศัยหลักการของการละลายอิ่มตัวของเกลือในน้ำที่อุณหภูมิต่างๆกัน โดยที่ความสามารถในการละลายแปรผันตรงกับอุณหภูมิของตัวทำละลาย ซึ่งเกลือที่จะนำมาใช้ในบ่อประเภทนี้ จำเป็นจะต้องมีอัตราการเปลี่่ยนแปลงความสามารถในการละลายต่ออุณหภูมิสูง เพื่อที่จะได้สามารถสร้าง gradient ของความเค็มต่อความลึกได้สูง และมีความสามารถในการเก็บความร้อนได้ดี
\subsection{แบตเตอรี่}
\label{sec:orgbcac65f}

\subsection{ล้อตุนกำลัง (Flywheel)}
\label{sec:org313ba94}

\section{การวิเคราะห์ต้นทุน}
\label{sec:orgebfffaf}

เคยสงสัยกันบ้างไหมว่า เวลาที่การไฟฟ้าเก็บค่าไฟเราหน่วยละ 3 บาทกว่าๆนั้น เขาคิดคำนวณกันมาอย่างไร มีหลักฐานอ้างอิงหรือข้อมูลอะไรมาช่วยสนับสนุนนี้ไหม หรือว่าแค่นั่งเทียนกำหนดเลขกลมๆขึ้นมา จริงๆแล้วก็คงไม่ใช่อย่างนั้น และแน่นอนว่าค่าไฟที่เก็บนั้นก็คงไม่ได้เท่าทุนพอดี คงจะต้องมีส่วนบวกเพื่อให้เป็นกำไรไว้ไม่มากก็น้อยเป็นแน่

ในบทนี้ เราจะมาพูดถึงการวิเคราะห์ต้นทุนทางเศรษฐศาสตร์ของการผลิตพลังงานเพื่อใช้ในการเปรียบเทียบระหว่างการผลิตไฟฟ้าปัจจุบัน (พ.ศ. 2560) โดยส่วนมากยังพึ่งพาเชื้อเพลิงปิโตรเลียมอยู่กับการผลิตไฟฟ้าจากพลังงานทดแทนซึ่งเราได้กล่าวถึงเทคโนโลโยีและอุปกรณ์ที่ต้องใช้ไปในส่วนที่ 1

หลายครั้งที่วิศวกรโดยเฉพาะอย่างยิ่งในสถานศึกษา (ตัวผมเองก็ด้วย) คิดวิเคราะห์ปัญหาทางพลังงานที่มีอยู่ในปัจจุบันโดยยังไม่ได้พิจารณาเรื่องของความเหมาะสมของเทคโนโลยีทางเศรษฐศาสตร์ หรือที่เรียกง่ายๆว่า เทคโนโลยีนั้นมันแพงเกินไปหรือเปล่า การจะพิจารณาความเป็นไปได้ที่จะนำเทคโนโลยีพลังงานหนึ่งๆมาใช้ แม้ว่าจะมีความล้ำสมัย สะอาด และเป็นมิตรต่อสิ่งแวดล้อมเพียงใด หากมีราคาแพงกว่าของเดิมที่ใช้อยู่ปัจจุบัน ก็คงยากที่จะโน้มน้าวให้ประชาชนส่วนมากเห็นดีเห็นงามไปด้วย ไม่ใช่ว่าพวกเขาไม่ได้รักโลก หรือไม่ห่วงเรื่องสิ่งแวดล้อม แต่ว่าการจะบอกว่าได้โปรดใช้ของที่แพงขึ้นหน่อยเพื่อให้โลกสะอาดขึ้นก็ฟังดูเป็นข้ออ้างที่อาจจะดูหลักลอยไปสักหน่อย วิธีง่ายที่สุดที่จะชวนให้ประชาชนทั่วไปหันมาสนใจการใช้พลังงานทดแทนอย่างจริงจังก็คือต้องบอกว่าของใหม่นั้น\textbf{ถูกกว่า}

ดังนั้น เพื่อจะแน่ใจว่าเทคโนโลยีพลังงานทดแทนของเรานั้นถูกกว่าไฟฟ้าที่ผลิตอยู่ปัจจุบัน เราจำเป็นจะต้องทำความเข้าใจก่อนว่าโครงสร้างต้นทุนการผลิตไฟฟ้า หรือพลังงานอื่นๆที่ใช้ในครัวเรือนปัจจุบันนั้นเป็นอย่างไร

\subsection{โครงสร้างต้นทุน\}}
\label{sec:org156caa2}

ศาสตร์เรื่องการวิเคราะห์โครงสร้างต้นทุนนั้นมีมานานโขอยู่ เริ่มจากปี \ldots{} ซึ่งพลังงานก็นับเป็นผลิตภัณฑ์อย่างหนึ่งซึ่งใช้สามารถจะวิเคราะห์ต้นทุนได้ การแบ่งประเภทต้นทุนนั้นสามารถทำได้อยู่หลายวิธี แล้วแต่จุดประสงค์และการนำไปใช้ประโยชน์ อย่างไรก็ดี ในหนังสือเล่มนี้เราต้องการศึกษาประเภทของต้นทุนเพื่อทำความเข้าใจแนวโน้มการเปลี่ยนแปลงเมื่อมีการพัฒนาเทคโนโลยีต่างๆที่เปลี่ยนไป จึงได้เลือกใช้วิธีการจำแนกต้นทุนตามความสัมพันธ์กับระดับของกิจกรรม ซึ่งสามารถสะท้อนความเปลี่ยนแปลงอันขึ้นอยู่กับระดับการผลิต โดยโครงสร้างต้นทุนแบบนี้สามารถแบ่งออกเป็นประเภทดังนี้

\begin{enumerate}
\item ต้นทุนคงที่ (Fixed Costs)

  เป็นต้นทุนส่วนที่ไม่มีการเปลี่ยนแปลงในช่วงระดับการผลิตหนึ่ง ซึ่งทำให้ต้นทุนต่อหน่วยลดลงเมื่อเพิ่มปริมาณการผลิตมากขึ้น
\item ต้นทุนผันแปร (Variable Costs)

  เป็นต้นทุนส่วนที่ต้นทุนรวมมีการเปลี่ยนแปลงขึ้นอยู่กับปริมาณการผลิต ในขณะที่ต้นทุนต่อหน่วยยังคงที่

\item ต้นทุนผสม (Mixed Costs)

  เป็นต้นทุนที่มีลักษณะของทั้งต้นทุนคงที่และผันแปรผสมกัน สามารถแบ่งได้เป็นสองประเภท

  \begin{enumerate}
  \item ต้นทุนกึ่งผันแปร (semi variable cost) เป็นต้นทุนที่จะมีส่วนหนึ่งคงที่ทุกระดับกิจกรรม และมีส่วนที่ผันแปรไปกับระดับกิจกรรม เช่น ค่าโทรศัพท์ เป็นต้น บางครั้งก็เป็นการยากที่จะประเมินส่วนที่คงที่หรือแปรผันของส่วนนี้
        \item ต้นทุนเชิงขั้น (step cost) หรือต้นทุนกึ่งคงที่ (semi fixed cost) หมายถึงต้นทุนที่คงที่ในช่วงระดับกิจกรรมหนึ่ง และเปลี่ยนไปคงที่ในอีกระดับกิจกรรมหนึ่ง เช่น ค่าผู้ควบคุมงาน เงินเดือน
    \end{enumerate}
\end{enumerate}

\subsection{มูลค่าเงินตามเวลา (Time Value of Money)\}}
\label{sec:org04636dc}

แนวคิดเรื่องของมูลค่าเงินตามเวลานั้นว่าด้วยมูลค่าของเงินที่เปลี่ยนแปลงไป ขึ้นอยู่กับเวลาที่เราได้รับหรือจ่ายเงินนั้นออกไป ฟังดูอาจจะแปลกๆอยู่สักหน่อย 100 บาทวันนี้ พรุ่งนี้ก็ยัง 100 บาทอยู่มิใช่หรือ แต่หากเริ่มเพิ่มเวลาเข้าไปเป็น 1 เดือน 1 ปี 10 ปี เงินนี้ก็อาจจะไม่เหมือนเดิมแล้ว พิจารณาได้อย่างง่ายด้วยคำถามนี้ หากมีคนสัญญาว่าจะให้เงินเรา 100 บาทตอนนี้เลยหรือ 100 บาทในอีก 10 ปีข้างหน้า ทุกคนคงตอบพร้อมเป็นเสียงเดียวกันว่า ขอเงิน 100 บาทตอนนี้เลยก็แล้วกัน นั่นเป็นเพราะว่าเงิน 100 บาทตอนนี้มี\textbf{มูลค่า}มากกว่าเงิน 100 บาทในอีก 10 ปีข้างหน้า

\subsection{ต้นทุนเฉลี่ยตลอดอายุโครงการ (Levelized Cost of Energy - LCOE)\}}
\label{sec:orgba134d4}

ในมุมมองของหน่วยงานควบคุมราคาหรือคุ้มครองผู้บริโภค ความสามารถในการทำกำไรหรืออัตราผลตอบแทนของโครงการโรงงานผลิตไฟฟ้าหนึ่งมักจะไม่ใช่สิ่งแรกที่น่าสนใจ ราคาต่อหน่วยพลังงานที่ผู้บริโภคจะต้องจ่ายเป็นตัววัดที่สามารถนำมาช่วยพิจารณาความเหมาะสมของการเลือกใช้พลังงานทางเลือกเพื่อผลิตไฟฟ้า

\begin{align}
  \label{eq:LCOE}
  \text{LCOE} &= \frac{\text{ผลรวมของต้นทุนที่พิจารณามูลค่าเงินตามเวลา}}{\text{ผลรวมของพลังงานไฟฟ้าที่ผลิตได้}} \\[10pt]
              &= \dfrac{\sum \dfrac{I_t + M_t + F_t}{(1+r)^t}}{\sum E_t} \\[10pt]
              &= \frac{\text{มูลค่าปัจจุบันสุทธิของต้นทุน}}{\text{พลังงานไฟฟ้าที่ผลิตได้ทั้งหมด}}
\end{align}

\subsection{อัตราผลตอบแทนภายใน (Internal Rate of Return - IRR)\}}
\label{sec:orgc05e71c}

\subsection{มูลค่าปัจจุบันสุทธิ (Net Present Value - NPV)}
\label{sec:org7c5ce8b}

\section{การพัฒนาอย่างยั่งยืน}
\label{sec:org41675d1}
\end{document}